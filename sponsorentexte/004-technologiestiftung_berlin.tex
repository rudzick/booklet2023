\section*{Platinsponsor und Aussteller}
\begin{center}
  \includegraphics[width=0.7\textwidth]{004_TSB_quer.png}
\end{center}
\noindent
Die {\bfseries Technologiestiftung Berlin} ist eine unabhängige und gemeinnützige Stiftung. Wir arbeiten für ein lebenswertes, smartes Berlin~-- und eine lebendige, transparente Stadtgesellschaft, die alle am digitalen Wandel teilhaben lässt. Mit digitalen Tools und smarten Lösungen tragen wir aktiv dazu bei, dass Berlin offen, nachhaltig und effizient wird. Viele unserer Projekte sind Leuchttürme, die beispielhaft die Chancen der Digitalisierung zeigen und Berlin über die Stadtgrenzen hinaus profilieren.

\noindent
Open Data und Open Source gehören zu unserer DNA. Gemeinsam mit Stadtgesellschaft, Verwaltung, Wissenschaft und Unternehmen nutzen wir das Potenzial offener Daten und Anwendungen, um Transparenz zu schaffen, Teilhabe zu ermöglichen und innovative Lösungen zu fördern. Und das in ganz unterschiedlichen Bereichen: Mit der Senatsverwaltung für Inneres, Digitalisierung und Sport bieten wir die \emph{Open Data Informationsstelle} an. Sie unterstützt die Berliner Verwaltung bei der Bereitstellung offener Daten und entwickelt darauf basierend eigene Anwendungen wie die Visualisierung der Berliner Haushaltsdaten oder das Organigramm-Tool. Im Projekt kulturdaten.berlin schaffen wir die erste offene, digitale Infrastruktur für Kulturschaffende und -institutionen in Berlin, gefördert durch die Senatsverwaltung für Kultur und Europa. Und die Open-Map-Anwendung \emph{Gieß den Kiez} vom CityLAB Berlin ermöglicht Bürger:innen die Pflege von über 800\,000 Stadtbäumen.

\noindent
Auf der FOSSGIS freuen wir uns auf den gemeinsamen Austausch und neue Impulse zu offenen Daten sowie Datenbank- und OpenStreetMap-Anwendungen!

