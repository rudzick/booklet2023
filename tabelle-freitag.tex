\newpage
\renewcommand{\arraystretch}{1.4}
\section*{Vorträge am Freitag}\label{freitag}
\renewcommand{\conferenceDay}{\freitag}
\setPageBackground
\small
\noindent\begin{tabular}{Z{0.7cm}Z{2.0cm}Z{2.0cm}Z{2.0cm}}
  & \multicolumn{1}{c}{\cellcolor{geoblau} Hörsaal 1 (0115)}
  & \multicolumn{1}{c}{\cellcolor{hellgelb} Hörsaal 2 (0110)}
  & \multicolumn{1}{c}{\cellcolor{hellgruen} Hörsaal 3 (0119)}
  \tabularnewline
09:00
  \talk{Das Beste der 60er, 70er und 80er: hochauflösende Spionagesatellitenaufnahmen}{Markus Metz}
  \talk{GeoNode in Forschungsdateninfrastrukturen}{Benedikt Gräler, Henning Bredel}
  \talk{Kartieren im Hochgebirge – ein Praxisbericht}{Mathias Gröbe}
  \tabularnewline
  09:30
  \talk{Automatisierte Detektion von Baumstandorten in der Metropole Ruhr}{Markus Metz, Anika Weinmann, Lina Krisztian}
  \talk{GIS und Datenströme: Stream Processing mit Apache StreamPipes}{Florian Micklich}
  \talk{Öffentliche Räume auf Basis von OSM-Daten kartieren}{Ester Scheck}
  \tabularnewline
  10:00
  \talk{\mbox{Prozessierung} von UAS-Befliegungen automatisiert gedacht}{Christian Bauer, Martin Weis}
  \talk{SIGNALO~--\linebreak Erhebung und Darstellung von Strassenschildern mit QGIS}{Isabel Kiefer}
  \talk{\mbox{Trinkwasser} und Trinkwasser-Orte-Mapping}{Annika}
  \tabularnewline
  \rowcolor{commongray}
  10:30 & \multicolumn{3}{c}{%
    \parbox[c]{24pt}{%
      \includegraphics[height=10pt]{cafe}%
    }
    Frühstückspause
  } \tabularnewline
\end{tabular}

\noindent\begin{tabular}{Z{0.7cm}Z{2.0cm}Z{2.0cm}Z{2.0cm}}
  & \multicolumn{1}{c}{\cellcolor{geoblau} Hörsaal 1 (0115)}
  & \multicolumn{1}{c}{\cellcolor{hellgelb} Hörsaal 2 (0110)}
  & \multicolumn{1}{c}{\cellcolor{hellgruen} Hörsaal 3 (0119)}
  \tabularnewline
11:00
  \talk{Ad-hoc-QGIS-Plugin-Entwicklung zur Bewertung der radiologischen Lage im Ukrainekrieg}{Marco Lechner}
  \talk{Lightning Talks}{}
  \talk{How far, how much, how many - Hilbert und Dijkstra zum Appell!}{Matthias Daues}
  \tabularnewline
  11:30
  \talk{Open Geodata, GI-Software und -Science am Beispiel einer räumlichen COVID-Studie}{Tobia Lakes}
  \talk{Irische Ogham-Steine in OSM und Wikidata}{Florian Thiery}
  \talk{Persistente \mbox{Identifikatoren} für Open-Source-GIS: Best\linebreak Practices und Bleeding Edge}{Peter Löwe, Ralf Löwner}
  \tabularnewline
 12:00
  \talk{Aufbau einer Plattform für die Risikobewertung von Biodiversitätsportfolios}{Markus Eichhorn}
  \talk{Strategie und Wunschzettel}{Roland Olbricht}
  \talk{Open Geodata and -software im Hochschulstudium der Geographie}{Tobia Lakes}
  \tabularnewline
  \rowcolor{commongray}
  12:30 & \multicolumn{3}{c}{%
    \parbox[c]{24pt}{%
      \includegraphics[height=10pt]{restaurant}%
    }
    Mittagspause
  } \tabularnewline
\end{tabular}


\noindent\begin{tabular}{Z{0.7cm}Z{3.0cm}Z{3.0cm}}
  & \multicolumn{1}{c}{\cellcolor{dezentrot} HS 4 | Demo (0313)}
  & \multicolumn{1}{c}{\cellcolor{audimax} Exp | Anw (1306)}
  \tabularnewline
11:00
  \talk{Sozialhelden: Wie barrierefrei ist unser Planet? \emph{(45 Min.)}}{Björn Uhlig,Sebastian Felix Zappe }
  \talk{Anwendertreffen 3D Tiles \emph{(45 Min.)}}{Pirmin Kalberer}
  \tabularnewline
12:00
  \talk{Lightning Talks}{}
  \talk{}{}
  \tabularnewline
  \rowcolor{commongray}
 12:30 & \multicolumn{2}{c}{%
    \parbox[c]{24pt}{%
      \includegraphics[height=10pt]{restaurant}%
    }
    Mittagspause
  } \tabularnewline
%\end{center}
\end{tabular}

\vspace{1.5\baselineskip}

\noindent\begin{tabular}{Z{0.7cm}Z{6.85cm}}
  & \multicolumn{1}{c}{\cellcolor{geoblau} Hörsaal 1 (0115)}
  \tabularnewline
  14:00
  \talk{16 Jahre FOSSGIS und OSM}{Jochen Topf}
  \tabularnewline
  14:30
  \talk{FOSSGIS-Jeopardy}{Johannes Kröger, Tobia Lakes}
  \tabularnewline
  15:00
  \talk{Abschlussveranstaltung}{FOSSGIS e.V.}
  \tabularnewline
  15:30
  \talk{Sektempfang}{FOSSGIS e.V.}
  \tabularnewline
\end{tabular}

\vspace{0.5\baselineskip}
\enlargethispage{1.0\baselineskip}
\renewcommand{\arraystretch}{1.4}
\noindent\begin{tabular}{Z{0.7cm}Z{2.0cm}Z{2.0cm}Z{2.0cm}}
  & \multicolumn{1}{c}{\cellcolor{commongray} Exkursion 1}
  & \multicolumn{1}{c}{\cellcolor{commongray} Exkursion 2}
  & \multicolumn{1}{c}{\cellcolor{commongray} Mapathon}
  \tabularnewline
  16:30
  \talk{Führung durch die Kartenabteilung der Staatsbibliothek zu Berlin}{}
  \talk{Führung über das WiSta-Gelände}{}
  \talk{Mapathon}{}
  \tabularnewline
    \rowcolor{commongray}
  19:30 & \multicolumn{3}{c}{%
    \parbox[c]{24pt}{%
      \includegraphics[height=10pt]{restaurant}%
    }
    OpenStreetMap-Vorabend
  } \tabularnewline
\end{tabular}
\renewcommand{\arraystretch}{1.0}
\normalsize

\newpage
