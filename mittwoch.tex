
% time: Wednesday 10:00
% URL: https://pretalx.com/fossgis2023/talk/fossgis2023-25280-erffnung/

%
\newTimeslot{10:00}
\noindent\abstractHSeins{%
  FOSSGIS e.V.%
}{%
  Eröffnung%
}{%
}{%
  Feierliche Eröffnung der Konferenz durch Vertreter des FOSSGIS e.\,V. mit wertvollen Hinweisen zum
  Ablauf und der Organisation.%
}%


%%%%%%%%%%%%%%%%%%%%%%%%%%%%%%%%%%%%%%%%%%%

% time: Wednesday 10:40
% URL: https://pretalx.com/fossgis2023/talk/fossgis2023-28211-die-berliner-opensource-strategie/

%
\newSmallTimeslot{10:40}
\noindent\abstractHSeins{%
StS Dr. Kleindiek%
}{%
  Die Berliner OpenSource Strategie%
}{%
}{%
  Zur Eröffnung der Konferenz wird der Leiter der Stabsstelle Digitalisierung Berlin etwas zur
  Berliner Open-Source-Strategie erzählen.%
}%
%%%%%%%%%%%%%%%%%%%%%%%%%%%%%%%%%%%%%%%%%%%

% time: Wednesday 11:00
% URL: https://pretalx.com/fossgis2023/talk/fossgis2023-23732-es-ist-doch-nur-software-warum-beschafft-die-ffentliche-hand-nicht-mehr-foss-/

%
\newSmallTimeslot{11:00}
\noindent\abstractHSeins{%
  Claus Wickinghoff, Miriam Seyffarth%
}{%
  Es ist doch nur Software~-- warum\linebreak beschafft die öffentliche Hand nicht mehr FOSS?%
}{%
}{%
  Die Beschaffung von Software durch die öffentliche Hand erfolgt oft
  durch Einsatz der EVB-IT. Diese sind jedoch für das Zusammenspiel mit
  Open-Source-Software nicht ideal geignet.
  In unserem Vortrag wollen wir die kritischen Punkte aufzeigen und
  erläutern, wie schon in der Ausschreibung die Beschaffung von Open
  Source Software für beide Seiten einfacher gehandhabt werden
  kann. Kombiniert wird dies mit einem Ausblick, wohin sich die
  öffentliche Beschaffung weiterentwickelt.%
}%


%%%%%%%%%%%%%%%%%%%%%%%%%%%%%%%%%%%%%%%%%%%

% time: Wednesday 11:30
% URL: https://pretalx.com/fossgis2023/talk/fossgis2023-23831-basemap-de-amtliche-geodaten-fr-deutschland/

%
\newTimeslot{11:30}
\noindent\abstractHSeins{%
  Arnulf Christl%
}{%
  basemap.de~-- amtliche Geodaten für Deutschland%
}{%
}{%
  basemap.de ist das Ergebnis der SmartMapping AG der AdV. Es ist ein Karten- und Datenwerk auf
  Basis einer Vielzahl von Datenquellen aus Bund und Ländern, u.\,a. Basis-DLM, DGM5, Hauskoordinaten,
  Zensus-Daten und vieles mehr. Die Daten werden in einer mit Open-Source-Software implementierten
  Architektur tagesaktuell automatisiert verarbeitet und als Vector-Tile-Archiv mit Styling,
  Rastertile-Diensten, GeoPackages (nur Open-Data-Länder) und mit dem selbstentwickelten
  basemap.de-viewer bereitgestellt.%
}%


%%%%%%%%%%%%%%%%%%%%%%%%%%%%%%%%%%%%%%%%%%%

% time: Wednesday 11:30
% URL: https://pretalx.com/fossgis2023/talk/fossgis2023-23415-ein-apache-superset-plugin-zur-interaktiven-datenreprsentation-mit-karten-und-charts/

%

\noindent\abstractHSzwei{%
  Sven Burbeck, Jan Suleiman, Christian Mayer%
}{%
  Ein Apache Superset Plugin zur\linebreak inter\-aktiven Datenrepräsentation\linebreak mit Karten und Charts%
}{%
}{%
  Der Vortrag beleuchtet die Verknüpfung unterschiedlicher Visualiserungsmethoden wie Karten,
  Diagramme und Dashboards anhand eines konkreten Umsetzungsprojektes~-- der Entwicklung eines
  Plugins für die Business-Intelligence-Software Apache Superset.
  Das Plugin dient zur Datenvisualisierung auf Basis der Kartenbibliothek OpenLayers und~-- darüber
  hinausgehend~-- der Einbindung von Charts der Bibliothek Apache ECharts als interaktive räumliche
  Objekte in OpenLayers.%
}%


%%%%%%%%%%%%%%%%%%%%%%%%%%%%%%%%%%%%%%%%%%%

% time: Wednesday 11:30
% URL: https://pretalx.com/fossgis2023/talk/fossgis2023-23413-introduction-to-coordinate-systems/

%

\noindent\abstractHSdrei{%
  Javier Jimenez Shaw%
}{%
  Introduction to Coordinate Systems%
}{%
}{%
  Brief introduction to Coordinate Reference Systems. Why do we need them; how do we model the earth
  (sphere, ellipsoid, geoid); Why/how do we use projected systems, like transverse Mercator or LCC.
  What is on EPSG catalog.%
}%


%%%%%%%%%%%%%%%%%%%%%%%%%%%%%%%%%%%%%%%%%%%

% time: Wednesday 12:00
% URL: https://pretalx.com/fossgis2023/talk/fossgis2023-23874-geodaten-als-hochwertige-datenstze/

%
\newSmallTimeslot{12:00}
\noindent\abstractHSeins{%
  Falk Zscheile%
}{%
  Geodaten als hochwertige\linebreak Datensätze%
}{%
}{%
  Die Public Sector Information Richtlinie (PSI RL) erklärt bestimmte Geodaten zu hochwertigen
  Datensätzen. Welche Geodaten hierzu gehören und wie sie zu lizenzieren sind, das kann die EU
  Kommission durch durch Kommissionsverordnung näher bestimmen.%
}%
\newpage

%%%%%%%%%%%%%%%%%%%%%%%%%%%%%%%%%%%%%%%%%%%

% time: Wednesday 12:00
% URL: https://pretalx.com/fossgis2023/talk/fossgis2023-23711-ein-web-dashboard-fr-besucherinformationssysteme-mittels-react-und-sensorthings-api/

%

\noindent\abstractHSzwei{%
  Mariyan Stamenov, Pascal Neis, Klaus Böhm, Cédric Roussel%
}{%
  Ein Web-Dashboard für Besucher\-informationssysteme mittels React und SensorThings API%
}{%
}{%
  In diesem Vortrag wird ein Web-Dashboard für ein Besucherinformationssystem vorgestellt, welches
  ausschließlich mit Open-Source-Bibliotheken wie React, Mobx-State-Tree, Leaflet, MaterialUI und
  Recharts implementiert wurde. Der so entstandene Prototyp kommt bereits am Campus der Hochschule
  Mainz zum Einsatz. Das dortige Gesamtsystem besteht aus verschiedenen Sensoren zur Messung der
  aktuellen Besucher, wobei die Datenverwaltung der Sensormesswerte über die OGC SensorThings API
  erfolgt.%
}%


%%%%%%%%%%%%%%%%%%%%%%%%%%%%%%%%%%%%%%%%%%%

% time: Wednesday 12:00
% URL: https://pretalx.com/fossgis2023/talk/fossgis2023-23895-geocoding-fr-einsteiger/

%

\noindent\abstractHSdrei{%
  Sarah Hoffmann%
}{%
  Geocoding für Einsteiger%
}{%
}{%
  Dieser Vortrag vermittelt Anfängern einen praktischen Einstieg in die Welt des
  Geocodings. Er stellt die verschiedenen Formen des Geocodings vor und zeigt, wie
  man in der eigenen Anwendung (webiste, QGIS, etc.) die Open-Source-Tools
  Nominatim und Photon zum Geocoding verwenden kann. Dazu gibt es Tipps und Tricks,
  wie man seine speziellen Suchprobleme noch effizienter lösen kann.%
}%


%%%%%%%%%%%%%%%%%%%%%%%%%%%%%%%%%%%%%%%%%%%

% time: Wednesday 14:00
% URL: https://pretalx.com/fossgis2023/talk/fossgis2023-23552-fragestunde-kartographie-mit-qgis/

%
\newTimeslot{14:00}
\noindent\abstractExpBoFAnw{%
  Mathias Gröbe, Johannes Kröger%
}{%
  Fragestunde Kartographie mit QGIS%
}{%
}{%
  Du wolltest schon immer wissen, wie du deine Daten besser visualisieren kannst? Wie Informationen
  am besten in Karten dargestellt werden können und das auch noch auf eine ästhetisch ansprechende
  Weise? Zwei Kartographen sollten da doch helfen können! Wir freuen uns auf den Austausch mit dir
  und auf deine Fragen. Fragen oder Themenwünsche können gerne schon im Vorfeld per E-Mail
  eingesandt werden.%
}%


%%%%%%%%%%%%%%%%%%%%%%%%%%%%%%%%%%%%%%%%%%%

% time: Wednesday 14:00
% URL: https://pretalx.com/fossgis2023/talk/fossgis2023-23823-openlayers-feature-frenzy/

%

\noindent\abstractHSeins{%
  Andreas Hocevar, Marc Jansen%
}{%
  OpenLayers Feature Frenzy%
}{%
}{%
  Vor über 10 Jahren war OpenLayers die erste Wahl unter den spärlich verfügbaren Web-Mapping-Bibliotheken.
  Ein Rewrite im Jahr 2012 und die zunehmende Verbreitung von Leaflet and Mapbox GL JS
  haben zu einem spürbaren Verlust an Popularität geführt. Heute bedient OpenLayers ein bisschen
  mehr als eine Nische als Mapping-Bibliothek mit den meisten Features und der größten Flexibilität bei
  guter Performance, die mit der Komplexität der an sie gestellten Aufgaben langfristig mithält.%
}%
\newpage

%%%%%%%%%%%%%%%%%%%%%%%%%%%%%%%%%%%%%%%%%%%

% time: Wednesday 14:00
% URL: https://pretalx.com/fossgis2023/talk/fossgis2023-23720-bestimmung-des-einflusses-von-pnv-verkehrsnetzen-auf-die-erreichbarkeit/

%

\noindent\abstractHSzwei{%
  Konstantin Geist%
}{%
  Bestimmung des Einflusses von ÖPNV-Verkehrsnetzen auf die\linebreak Erreichbarkeit%
}{%
}{%
  Im Kontext des Forschungsprojektes RAFVINIERT, welches sich mit der Versorgung von Seniorinnen und
  Senioren im ländlichen Raum beschäftigt, wird dem Faktor ÖPNV eine große Bedeutung für die
  Erreichbarkeit von seniorenrelevanten Einrichtungen attestiert. Durch räumliche Analysen und
  Routing-Algorithmen wurde die Versorgungssituation von verschiedenen Standorten modelliert und
  eine Vergleichbarkeit zu fußläufigen Erreichbarkeiten hergestellt.%
}%


%%%%%%%%%%%%%%%%%%%%%%%%%%%%%%%%%%%%%%%%%%%

% time: Wednesday 14:00
% URL: https://pretalx.com/fossgis2023/talk/fossgis2023-23914-eine-schnittstelle-zur-lnderbergreifenden-bereitstellung-von-alkis-daten/

%

\noindent\abstractHSdrei{%
  Christopher Frank, Alexander Willner%
}{%
  Eine Schnittstelle zur länderübergreifenden Bereitstellung von\linebreak ALKIS-Daten%
}{%
}{%
  ALKIS ist ein föderalistisch geprägter Datensatz. Mit der Länderhoheit über die Erfassung, Pflege
  und Abgabe der Daten entwickelte sich in Deutschland ein heterogener Datenbestand. Die Beschaffung
  und Integration erweist sich insbesondere bei länderübergreifenden Projekten als herausfordernd.
  Im Projekt ADA wird der Beschaffungsprozess durch die Entwicklung einer Shop-Schnittstelle zur
  verbesserten Auffindbarkeit und zur Direktintegration länderübergreifender Daten in Drittsysteme
  optimiert.%
}%


%%%%%%%%%%%%%%%%%%%%%%%%%%%%%%%%%%%%%%%%%%%

% time: Wednesday 14:00
% URL: https://pretalx.com/fossgis2023/talk/fossgis2023-23882-qgis-slim-wir-entschlacken-die-qgis-oberflche/

%

\noindent\abstractHSvier{%
  Reinhold Stahlmann%
}{%
  QGIS Slim: Wir entschlacken die QGIS-Oberfläche%
}{%
}{%
  Fokussiert und bedarfsgerecht: Wir passen die Oberfläche von QGIS auf unsere Bedürfnisse an, um
  Platz auf mobilen Endgeräten zu sparen und die Übersichtlichkeit zu erhöhen. Wir erstellen
  verschiedene GUIs für Datenerfassung im Feld und Büro und gehen auch auf fertige Lösungen wie
  QField oder IntraMaps ein.%
}%
\vspace{1.92\baselineskip}
\sponsorBoxA{401_komoot.png}{0.45\textwidth}{4}{%
\textbf{Bronzesponsor}\\
\noindent\small Mit der Komoot-App und ihrem Routenplaner kannst du neue Abenteuer leicht finden, planen und teilen. Mit Leidenschaft fürs Entdecken und den besten Empfehlungen der Community ist es die Mission von Komoot, einzigartige Abenteuer für alle zu ermöglichen.
\normalsize
}%
%

%%%%%%%%%%%%%%%%%%%%%%%%%%%%%%%%%%%%%%%%%%%

% time: Wednesday 14:30
% URL: https://pretalx.com/fossgis2023/talk/fossgis2023-23792-performantes-rendering-von-vektordaten-neuigkeiten-aus-der-webgl-openlayers-library/

%
\newTimeslot{14:30}
\noindent\abstractHSeins{%
  Andreas Jobst%
}{%
  Performantes Rendering von Vektordaten~-- Neuigkeiten aus der WebGL-OpenLayers-Library%
}{%
}{%
  Unterstützt durch WebGL können Vektordaten in OpenLayers nun sehr performant gerendert werden. Wir
  geben Einblick in die aktuellen Entwicklungen in dieser neuen OpenLayers-Bibliothek.%
}%


%%%%%%%%%%%%%%%%%%%%%%%%%%%%%%%%%%%%%%%%%%%

% time: Wednesday 14:30
% URL: https://pretalx.com/fossgis2023/talk/fossgis2023-23566-ein-offener-ansatz-zur-ermittlung-von-verkehrsaufkommen-anhand-des-internet-of-things/

%

\noindent\abstractHSzwei{%
  Sarah Schütz, Pascal Neis%
}{%
  Ein Offener Ansatz zur Ermittlung von Verkehrsaufkommen anhand\linebreak des Internet of Things%
}{%
}{%
  Im Vortrag wird gezeigt, wie unter Verwendung des IoT der Verkehrsfluss mittels Bilderkennung und
  frei verfügbaren Webcams bestimmt und über offene, standardisierte Schnittstellen bereitgestellt
  und visualisiert werden kann. Bei dem entwickelten Workflow kommen dabei ausschließlich
  Open-Source-Bibliotheken und Dienste sowie Open Data zum Einsatz.%
}%
\newpage

%%%%%%%%%%%%%%%%%%%%%%%%%%%%%%%%%%%%%%%%%%%

% time: Wednesday 14:30
% URL: https://pretalx.com/fossgis2023/talk/fossgis2023-23838-reality-check-offene-katasterdaten-in-deutschland/

%

\noindent\abstractHSdrei{%
  Julia Wielgosch, Marina Happ%
}{%
  Reality Check: Offene\linebreak Katasterdaten in Deutschland%
}{%
}{%
  Die Studie des Wissenschaftlichen Instituts für Infrastruktur und Kommunikationsdienste (WIK)
  untersucht den aktuellen Stand offener Katasterdaten in Deutschland. Sie zeigt, dass die
  Potenziale offener Katasterdaten noch nicht umfänglich ausgeschöpft werden können, da die Daten
  nicht in allen Bundesländern kostenfrei sind und zudem in unterschiedlichen Formaten und
  Qualitäten vorliegen. Für datenbereitstellende Behörden ist der Schritt zu Open Data mit
  finanziellen Herausforderungen verbunden.%
}%
\vspace{1.92\baselineskip}
\sponsorBoxA{409_gkg_logo.png}{0.32\textwidth}{6}{%
\textbf{Bronzesponsor}\\
\noindent Schulungen, Dienstleistungen und Support rund um QGIS, GRASS, SpatiaLite und PostGIS. Mit angepassten QGIS-Oberflächen, dem QGIS-Modeller sowie SpatialSQL strukturiere ich Ihre Geodaten und ermögliche maßgeschneiderte Analyse und Präsentation mit freier GIS-Software. http://www.gkg-kassel.de
}%
%

%%%%%%%%%%%%%%%%%%%%%%%%%%%%%%%%%%%%%%%%%%%

% time: Wednesday 15:00
% URL: https://pretalx.com/fossgis2023/talk/fossgis2023-23903-leaflet-webmaps-erstellen-leicht-gemacht/

%
\newTimeslot{15:00}
\noindent\abstractHSeins{%
  Numa Gremling%
}{%
  Leaflet: Webmaps erstellen leicht\linebreak gemacht%
}{%
}{%
  Leaflet ist eine der meistbenutzten und beliebtesten Bibliotheken um Webmaps zu erstellen. In
  diesem Vortrag lernen Sie warum.%
}%


%%%%%%%%%%%%%%%%%%%%%%%%%%%%%%%%%%%%%%%%%%%

% time: Wednesday 15:00
% URL: https://pretalx.com/fossgis2023/talk/fossgis2023-23853-freie-fahrt-fr-die-mobilittswende-ergebnisse-explorativer-analysen-freier-geodaten/

%

\noindent\abstractHSzwei{%
  Susanne Schröder-Bergen, Dominik Kremer%
}{%
  Freie Fahrt für die Mobilitätswende? Ergebnisse explorativer Analysen freier Geodaten%
}{%
}{%
  Seit einiger Zeit wird in Politik, Planung und Gesellschaft verstärkt die Frage diskutiert, wie
  Mobilität (in Deutschland) in Zukunft gedacht werden kann. Auf der Grundlage von OSM und anderen
  frei verfügbaren Geodaten präsentieren wir eine explorative ortsbezogene Geomodellierung und -analyse,
  um verschiedene Perspektiven auf Mobilität im Kontext der Mobilitätswende zu ergründen,
  die über traditionelle GIS-Systeme hinausgehen.%
}%
\newpage

%%%%%%%%%%%%%%%%%%%%%%%%%%%%%%%%%%%%%%%%%%%

% time: Wednesday 15:00
% URL: https://pretalx.com/fossgis2023/talk/fossgis2023-23918-alkis-daten-2022-zur-zugnglichkeit-und-umzugnglichkeit-einer-infrastruktur/

%

\noindent\abstractHSdrei{%
  Claas Leiner%
}{%
  ALKIS-Daten 2022~--  Zur Zugänglichkeit und Umzugänglichkeit einer Infra\-struktur%
}{%
}{%
  Die Beschaffung von ALKIS-Daten für den Breitband-Netzausbau ist eine recht sperrige Aufgabe.
  Die Länder gehen sehr unterschiedliche Wege, um die Daten verfügbar oder auch möglichst
  unverfügbar zu machen. Als freiberuflicher Geodatendienstleister, der im Jahr 2022 umfangreiche
  Beschaffungen für ein Telekommunikationsunternehmen umgesetzt hat, berichte ich von den aktuellen
  Heruasforderungen.%
}%


%%%%%%%%%%%%%%%%%%%%%%%%%%%%%%%%%%%%%%%%%%%

% time: Wednesday 15:00
% URL: https://pretalx.com/fossgis2023/talk/fossgis2023-27094-spontane-lightning-talks/

%

\noindent\abstractHSvier{%
  FOSSGIS e.V.%
}{%
  Spontane Lightning Talks%
}{%
}{%
  Spontan eingereichte Lightning Talks. Jeder kann einen Vortrag halten, Registrierung erfolgt am
  FOSSGIS-Stand.%
}%

%%%%%%%%%%%%%%%%%%%%%%%%%%%%%%%%%%%%%%%%%%%

% time: Wednesday 16:00
% URL: https://pretalx.com/fossgis2023/talk/fossgis2023-23694-anwendertreffen-lizmap-webclient/

%
\newTimeslot{16:00}
\noindent\abstractExpBoFAnw{%
  Günter Wagner%
}{%
  Anwendertreffen Lizmap-Webclient%
}{%
}{%
  Die deutschsprachige Anwendergruppe für den WebClient Lizmap möchte das Treffen zum
  Erfahrungsaustausch nutzen.
  Teilnehmer können ihre eigenen, mit Lizmap realisierten, WebGIS-Projekte vorstellen. Ferner kann
  über aktuelle Fragen/Probleme und zukünftige, gewünschte Erweiterungen in Lizmap diskutiert
  werden.
  Das Anwendertreffen richtet sich sowohl an neu Interessierte, als auch an Anwender, die bereits
  mit Lizmap arbeiten.%
}%


%%%%%%%%%%%%%%%%%%%%%%%%%%%%%%%%%%%%%%%%%%%

% time: Wednesday 16:00
% URL: https://pretalx.com/fossgis2023/talk/fossgis2023-23479-topdeutschland-2-0-portables-qgis-inklusive-offline-geobasisdaten/

%

\noindent\abstractHSeins{%
  Anna-Lena Hock, Christoph Welker%
}{%
  TopDeutschland 2.0~-- Portables QGIS inklusive Offline-Geobasisdaten%
}{%
}{%
  Der Vortrag stellt den Funktionsumfang, sowie die technische Umsetzung der TopDeutschland 2.0 vor.
  Die TopDeutschland ist eine Anwendung, welche auf Basis von QGIS ohne weitere Installation
  gestartet werden kann. Sie enthält deutschlandweite Geobasisdaten für den Offline-Zugriff, wobei
  MapProxy zur performanten Darstellung von Rasterdaten verwendet wird. Die Entwicklung der
  Anwendung ist eine Zusammenarbeit zwischen dem BKG, der WhereGroup sowie der norBIT GmbH.%
}%
\pagebreak

%%%%%%%%%%%%%%%%%%%%%%%%%%%%%%%%%%%%%%%%%%%

% time: Wednesday 16:00
% URL: https://pretalx.com/fossgis2023/talk/fossgis2023-23828-neues-vom-geostyler/

%

\noindent\abstractHSzwei{%
  Daniel Koch, Jan Suleiman%
}{%
  Neues vom GeoStyler%
}{%
}{%
  GeoStyler ist eine Open-Sourcer-JavaScript-Bibliothek zum Konvertieren zwischen Styling-Formaten
  und zur einfachen Erstellung von modernen Web-Oberflächen zum kartographischen Stylen von
  Geodaten. In diesem Vortrag werden die jüngsten Entwicklungen aus dem Projekt vorgestellt.%
}%


%%%%%%%%%%%%%%%%%%%%%%%%%%%%%%%%%%%%%%%%%%%

% time: Wednesday 16:00
% URL: https://pretalx.com/fossgis2023/talk/fossgis2023-23851-gebudedetektion-auf-basis-von-luftbildern-und-punktwolken-des-regionalverbands-ruhr/

%

\noindent\abstractHSdrei{%
  Leonie Krelaus, Inga-Mareike Nießen%
}{%
  Gebäudedetektion auf Basis von Luftbildern und Punktwolken des Regionalverbands Ruhr%
}{%
}{%
  Der Regionalverband Ruhr (RVR) möchte seinen Mitgliedern den Gebäudebestand im Verbandsgebiet
  bereitstellen. Die mundialis GmbH \& Co. KG entwickelte im Auftrag des RVR ein Verfahren zur
  möglichst automatisierten Gebäudedetektion und Veränderungen im Raum. Es wurde ein Workflow unter
  Einsatz der Software GRASS GIS (Linux-Ubuntu-Distribution) entwickelt. Das Ergebnis ist ein
  Vektordatensatz, der u. a. auch Informationen zur Anzahl der Stockwerke des erkannten Gebäudes
  beinhaltet.%
}%
\pagebreak

%%%%%%%%%%%%%%%%%%%%%%%%%%%%%%%%%%%%%%%%%%%

% time: Wednesday 16:00
% URL: https://pretalx.com/fossgis2023/talk/fossgis2023-23556-kartographische-generalisierung-mit-postgresql-und-postgis/

%

\noindent\abstractHSvier{%
  Mathias Gröbe, Robert Klemm%
}{%
  Kartographische Generalisierung mit PostgreSQL und PostGIS%
}{%
}{%
  PostgreSQL bildet mit seiner räumlichen Erweiterung PostGIS die Grundlage für viele Webkarten und
  eignet sich hervorragend für die Verarbeitung von großen Datensätzen und damit für die
  Generalisierung. Anhand von OpenStreetMap-Daten sollen Ansätze für das Zusammenfassen und
  Vereinfachen von Flächengeometrien, zum Ausdünnen von Straßennetzwerken sowie die Verdrängung von
  Objekten vorgestellt werden.%
}%
\vspace{1.92\baselineskip}
\small
\sponsorBoxA{403_sp_fossgis_2023.png}{0.45\textwidth}{4}{%
\textbf{Bronzesponsor und Aussteller}\\
\noindent {\bfseries Sourcepole AG} betreibt die leistungsfähige Web-GIS-Plattform \mbox{qgiscloud.com} zur Publikation von Karten, Daten und Diensten im Internet. Mit QGIS Cloud erhalten Sie eine vollständige Geodateninfrastruktur bestehend aus PostGIS-Data-Warehouse, QGIS-Server Webserver und QWC2 als Webclient.
}%
\normalsize
%

%%%%%%%%%%%%%%%%%%%%%%%%%%%%%%%%%%%%%%%%%%%

% time: Wednesday 16:30
% URL: https://pretalx.com/fossgis2023/talk/fossgis2023-23900-mobiles-gis-fr-das-landwirtschaftliche-versuchswesen/

%
\newTimeslot{16:30}
\enlargethispage{3\baselineskip}

\noindent\abstractHSeins{%
  Martin Weis, Christian Bauer%
}{%
  Mobiles GIS für das landwirtschaftliche Versuchswesen%
}{%
}{%
  Feldversuche erfordern von der Planung bis zur Umsetzung Geodaten, die im Büro und im Feld mit
  RTK-GPS genutzt werden. Wir zeigen die genutzten Open-Source-Werkzeuge sowie Datenflüsse und
  Schnittstellen.%
}%


%%%%%%%%%%%%%%%%%%%%%%%%%%%%%%%%%%%%%%%%%%%

% time: Wednesday 16:30
% URL: https://pretalx.com/fossgis2023/talk/fossgis2023-23844-mapcomponents-aktuelle-entwicklungen-des-react-komponenten-frameworks/

%

\noindent\abstractHSzwei{%
  Olaf Knopp%
}{%
  MapComponents~-- Aktuelle Entwicklungen des React-Komponenten-Frameworks%
}{%
}{%
  Das GIS-Komponenten-Framework MapComponents wurde Ende 2021 als Open-Source-Projekt auf Github
  veröffentlicht. In den letzten Monaten hat sich viel getan. Neben einer Vielzahl neuer Konponenten
  mit spannenden Funktionen wurden bereits erste Projekte mit MapComponents umgesetzt. Der Vortrag
  stellt das Projekt und die aktuellsten Entwicklungen und Features vor. Außerdem gibt er einen
  Ausblick auf die Backend-Komponenten, die mit Python umgesetzt werden.%
}%
\pagebreak

%%%%%%%%%%%%%%%%%%%%%%%%%%%%%%%%%%%%%%%%%%%

% time: Wednesday 16:30
% URL: https://pretalx.com/fossgis2023/talk/fossgis2023-23841-vollstndigkeitsabschtzung-von-gebuden-in-osm-mit-dem-global-human-settlement-layer/

%

\noindent\abstractHSdrei{%
  Laurens Oostwegel%
}{%
  Vollständigkeitsabschätzung von\linebreak Gebäuden in OSM mit dem Global\linebreak Human Settlement Layer%
}{%
}{%
  Die Qualitätsbewertung, inklusive Vollständigkeitsabschätzung, ist ein bedeutender Prozess bei
  Umgang mit Volunteered Geographic Information. Die Vollständigkeit von Gebäuden in OSM wird durch
  Vergleich mit dem Global Human Settlement Layer (GHSL) abgeschätzt. Hierfür wird der GHSL
  auf Quadtree-Kacheln projiziert und mit der entsprechenden Größe der Gebäudepolygone in OSM
  verglichen. Mit den Minutely Diffs von OSM bietet das System ein aktuelle und weltumspannende
  Vollständigkeitsabschätzung.%
}%
%\vspace{1.0\baselineskip}
\small
\sponsorBoxA{202_schneider-digital-logo-color-rgb.png}{0.35\textwidth}{3}{%
\textbf{Silbersponsor und Aussteller}
\vspace{0.5\baselineskip}

 \noindent {\bfseries Schneider Digital} Professional 3D \& VR/AR Hardware-Creator für Performance in Berechnung und Visualisierung.
Unser Produktportfolio: High Resolution 4K/8K-Monitore (UHD), 3D-Stereo- und Touch-Monitore von 22 bis 100 Zoll, VR/AR-Lösungen, vom Desktop-System bis hin zu Multi-Display-Walls.
Schneider Digital ist Hersteller der eigenen Powerwall-Lösung Laser smartVR-Wall sowie des passiven 3D-Stereo-Monitors und Desktop-VR-Systems 3D PluraView. Eigenentwickelte Performance"=Workstations mit Profi-Grafikkarten von AMD und NVIDIA sowie innovative Hardware-Peripherie (Tracking, 3D-Controller und Eingabegeräte u.\,v.\,a.) komplettieren das Angebot zu ganzheitlichen Arbeitsplatz-Lösungen für alle anspruchsvollen Einsatzbereiche.
}%
\normalsize
%

%%%%%%%%%%%%%%%%%%%%%%%%%%%%%%%%%%%%%%%%%%%

% time: Wednesday 17:00
% URL: https://pretalx.com/fossgis2023/talk/fossgis2023-23858-forstliche-standortbewertung-mit-qgis-grass-gis-und-qfield/

%
\newTimeslot{17:00}
\enlargethispage{3\baselineskip}
\noindent\abstractHSeins{%
  Klaus Mithöfer, Rüdiger Müller%
}{%
  Forstliche Standortbewertung mit QGIS, GRASS GIS und QField%
}{%
}{%
  Die Aufforstung von Waldflächen erfordert vertiefte Kenntnisse in der forstlichen
  Standortbewertung. Neben bodenkundlichen Eigenschaften lassen sich viele Standorteigenschaften wie
  Lage und Morphologie aus Oberflächenmodellen ableiten. In dem hier verfolgten Ansatz werden
  hochauflösende DGM mit GRASS GIS analysiert und die Ergebnisse für Parzellen klassifiziert. Die
  Ergebnisse werden in QGIS fachlich um weitere Informationen ergänzt und für die Bewertung im
  Gelände an QField übergeben.%
}%


%%%%%%%%%%%%%%%%%%%%%%%%%%%%%%%%%%%%%%%%%%%

% time: Wednesday 17:00
% URL: https://pretalx.com/fossgis2023/talk/fossgis2023-23836-qgis-web-client-2-qwc2-neues-aus-dem-projekt/

%

\noindent\abstractHSzwei{%
  Horst Düster, Sandro Mani%
}{%
  QGIS Web Client 2 (QWC2)~-- Neues\linebreak aus dem Projekt%
}{%
}{%
  Dieser Vortrag stellt den QWC2 vor und zeigt, wie einfach es ist, eigene QGIS-Projekte im Web zu
  veröffentlichen. Es wird ein Überblick über die QWC2-Architektur gegeben. Dabei ist es auch eine
  Gelegenheit, die letzten neuen Funktionen, die im letzten Jahr entwickelt wurden, und die Ideen
  für zukünftige Verbesserungen zu entdecken.%
}%
\pagebreak
\enlargethispage{2.0\baselineskip}

%%%%%%%%%%%%%%%%%%%%%%%%%%%%%%%%%%%%%%%%%%%

% time: Wednesday 17:00
% URL: https://pretalx.com/fossgis2023/talk/fossgis2023-23821-erdbeben-und-openstreetmap/

%
\renewcommand{\conferenceDay}{\mittwoch}
\setPageBackground

\noindent\abstractHSdrei{%
  Danijel Schorlemmer%
}{%
  Erdbeben und OpenStreetMap%
}{%
}{%
  Zur Berechnung der möglichen Folgen von Erdbeben ist die Kenntnis der Lage, der Grösse und des
  Typs von Gebäuden, ihr Wiederbeschaffungswert und die Anzahl der Bewohner nach Tageszeit
  notwendig. Mithilfe von OpenStreetMap und weiteren offenen Daten (z.B. länder-/region-spezifische
  Gebäudetypen, Global Human Settlement Layer) erzeugen wir ein globales, dynamisches,
  algorithmisches, und reproduzierbares Expositionsmodell zur probabilistischen Beschreibung dieser
  Parameter für jedes Gebäude der Welt.%
}%


%%%%%%%%%%%%%%%%%%%%%%%%%%%%%%%%%%%%%%%%%%%

% time: Wednesday 17:00
% URL: https://pretalx.com/fossgis2023/talk/fossgis2023-23907-gis-analysen-im-browser/

%

\noindent\abstractHSvier{%
  Numa Gremling%
}{%
  GIS-Analysen im Browser%
}{%
}{%
  GIS-Analysen sind direkt in einem Browser möglich und zwar ohne Anbindung an einen Server.
  Durchgesetzt hat sich vor allem die Bibliothek Turf.js.%
}%
%\pagebreak

%%%%%%%%%%%%%%%%%%%%%%%%%%%%%%%%%%%%%%%%%%%

% time: Wednesday 17:30
% URL: https://pretalx.com/fossgis2023/talk/fossgis2023-27669-postnas-suite-anwendertreffen/

%
\newSmallTimeslot{17:30}
\noindent\abstractExpBoFAnw{%
  %
}{%
  Anwendertreffen PostNAS-Suite%
}{%
}{%
  Die Anwender:innen der PostNAS-Suite kommunizieren über die Mailingliste und treffen sich zum
  Austausch. Die Zeit wird genutzt, um die mit der Umstellung auf die GeoInfoDok~7
  anstehenden Änderungen durchzusprechen. Einsteiger und Interessierte sind willkommen. Fragen können durch zahlreiche
  PostNAS-Suite-Anwender beantwortet werden.%
}%
\vspace{1.92\baselineskip}
\small
\sponsorBoxA{201_terrestris.png}{0.45\textwidth}{6}{%
\textbf{Silbersponsor und Aussteller}%
\vspace{0.5\baselineskip}

\noindent
{\bfseries terrestris GmbH \& Co. KG} ist Dienstleister für maßgeschneiderte Geoinformations-Lösungen mit Freier und Open-Source-Software. Wir entwickeln Lösungen, die den tatsächlichen Anforderungen unserer Kunden entsprechen. Dies ist nach unserem eigenen Verständnis auch Grundlage unserer fairen, transparenten und häufig langfristigen Kundenbeziehungen. Dabei decken wir das gesamte Spektrum von Beratung, Konzeptionierung, Entwicklung bis hin zu Wartung und Support an. Unser Leistungsspektrum erstreckt sich auf folgende Bereiche:
Geoportale und Stadtplandienste;
Fachanwendungen für unterschiedlichste Branchen;
Visualisierung von geographischen 3D-Daten im Browser;
Einsatz und Verwendung freier Geodaten wie OpenStreetMap;
QGIS.
}%
\normalsize
%

\vspace{1.0\baselineskip}
\sponsorBoxA{402_mundialis.png}{0.25\textwidth}{5}{%
\textbf{Bronzesponsor}\\
\noindent mundialis ist spezialisiert auf die Auswertung und Verarbeitung von Fernerkundungs- und Geodaten mit dem Schwerpunkt Cloud-basierte Geoprozessierung. Wir setzen freie und Open-Source-Geoinformationssysteme (GRASS GIS, actinia, QGIS, u.a.) ein, mit denen wir maßgeschneiderte Lösungen für den Kunden entwickeln.
}%
%
%%%%%%%%%%%%%%%%%%%%%%%%%%%%%%%%%%%%%%%%%%%


\renewcommand{\conferenceDay}{\mittwoch}
\setPageBackground
