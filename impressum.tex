\newpage
\section*{Impressum}
\label{impressum}
\pagestyle{cropmarksstyle}

\RaggedRight
{\small
Die FOSSGIS 2023 wird gemeinsam vom FOSSGIS e.V. und der Humboldt-Universität zu Berlin organisiert.

\vspace{0.5em}
\newlength\logoHeight
\setlength{\logoHeight}{4.0\baselineskip}
\begin{minipage}[c]{0.5\textwidth}
  \includegraphics[height=\logoHeight]{FOSSGIS}
\end{minipage}\\
\begin{minipage}[c]{0.35\textwidth}
  \includegraphics[height=\logoHeight]{HU_plus_Geographie_Logo}
\end{minipage}

\vspace{0.5em}
\noindent Verantwortlich für den Inhalt:\\
FOSSGIS e.V.\\
Bundesallee 23\\
10717 Berlin

\vspace{0.5em}
\noindent Diese Programmheft wurde unter Verwendung von Lua\LaTeX\ und 
anderer freier Software zusammengestellt.\\
Quellcode: github.com/rudzick/booklet2023\\
\noindent Satz und Layout: Oliver Rudzick, Michael Reichert\\
Artwork und Titelgestaltung: Dirk Schroer\\
Campusplan und Raumpläne: Benjamin Jakimow\\
Kartographie: OpenStreetMap, QGIS\\
Geodaten: \includegraphics[height=7pt]{copyright}~Open\-Street\-Map-Mitwirkende (Open Database License 1.0)\\
Icons in den Tabellen: SJJB Management, CC-0\\
Lektorat und Endkontrolle: Katja Haferkorn

\vspace{1em}
\noindent \begin{minipage}[htbp]{0.2\textwidth}
\noindent\includegraphics[width=\linewidth]{CC_BY-SA_icon}
\end{minipage}
\hfill
\begin{minipage}[hbtp]{0.74\textwidth}\RaggedRight
  {\small
    Alle Inhalte dieses Programmhefts unterliegen, sofern nicht anders angegeben,
    der Lizenz \emph{Creative Commons Namensnennung Weitergabe unter gleichen Bedingungen 3.0}.
    Logos von Firmen und Organisationen sind hiervon ausgenommen.
  }
\end{minipage}

\newpage
\pagestyle{page-uebersicht-adlershof}
\label{kartenseiten}
\null

\newpage
\pagestyle{page-esz}
\null

\newpage
\pagestyle{page-gi}
\null
