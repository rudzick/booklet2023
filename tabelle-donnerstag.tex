\newpage
\renewcommand{\arraystretch}{1.4}
\section*{Vorträge am Donnerstag\linebreak \small{\normalfont Übersicht über Anwendertreffen und Demo-Sessions auf Seite\,\pageref{demodonnerstag}}}\label{donnerstag}
\renewcommand{\conferenceDay}{\donnerstag}
\setPageBackground
%\small
\noindent\begin{tabular}{Z{0.7cm}Z{2.0cm}Z{2.0cm}Z{2.0cm}}
  & \multicolumn{1}{c}{\cellcolor{geoblau} Hörsaal 1 (0115)}
  & \multicolumn{1}{c}{\cellcolor{hellgelb} Hörsaal 2 (0110)}
  & \multicolumn{1}{c}{\cellcolor{hellgruen} Hörsaal 3 (0119)}
  \tabularnewline
09:00
  \talk{20 Jahre QGIS}{Marco Bernasocchi}
  \talk{XPlan-Reader~-- ein QGIS-Plugin}{Michael Stein}
  \talk{Neues von \mbox{actinia}}{Carmen Tawalika}
  \tabularnewline
  09:30
  \talk{Seekarten mit QGIS~-- geht das?}{Ulrike Assmann, Annette Hey}
  \talk{XPlanung mit Open-Source-Software}{Torsten Friebe}
  \talk{GeoServer-Cloud~-- ein Projektupdate und Erfahrungsberichte aus produktiven Umgebungen}{Andrea Borghi}
  \tabularnewline
  10:00
  \talk{QFieldCloud~-- Effiziente Zusammenarbeit im Feld}{Marco Bernasocchi}
  \talk{KomMonitor~-- kommunales Monitoring zur Raumentwicklung}{Sebastian Drost, Christian Danowski-Buhren}
  \talk{Neues aus dem GeoNode-Projekt}{Florian Hoedt}
  \tabularnewline
  \rowcolor{commongray}
  10:30 & \multicolumn{3}{c}{%
    \parbox[c]{24pt}{%
      \includegraphics[height=10pt]{cafe}%
    }
    Frühstückspause
  } \tabularnewline
\end{tabular}

\newpage
\noindent\begin{tabular}{Z{0.7cm}Z{2.0cm}Z{2.0cm}Z{2.0cm}}
  & \multicolumn{1}{c}{\cellcolor{geoblau} Hörsaal 1 (0115)}
  & \multicolumn{1}{c}{\cellcolor{hellgelb} Hörsaal 2 (0110)}
  & \multicolumn{1}{c}{\cellcolor{hellgruen} Hörsaal 3 (0119)}
  \tabularnewline
11:00
  \talk{Ablösung proprietärer Kartografie-Software durch eine Open-Source- und Open-Data-Lösung}{Robert Klemm, Luise Leffmann}
  \talk{Schnelle, flexible Volltextsuche in OpenStreetMap}{Wolfram Schneider}
  \talk{QField News: \mbox{Navigations-,} Profil- und Ausstecktool, QR-Codes, iOS und vieles mehr}{Matthias Kuhn}
  \tabularnewline
  11:30
  \talk{Vollständige Beschriftung in einem QGIS-Stadtplanprojekt mit Hilfe der Maplex-Label-E}{Larissa Bitterich}
  \talk{Vektortile-Erfahrungen mit Shortbread}{Michael Reichert}
  \talk{Mobile Erfassung von Einfahrten mit QGis und \mbox{merginmaps}}{Lars Lingner}
  \tabularnewline
 12:00
  \talk{Gelände- und Kartierpraktikum zur Erhebung von Barriereinformationen}{Christian Willmes}
  \talk{Generalisierung von OSM-Daten mit osm2pgsql}{Jochen Topf}
  \talk{QGIS und PostGIS nebst QField und QGIS-Server im Einsatz bei der Entsorgung Dortmund}{Jörg Thomsen, Jakob Kopec}
  \tabularnewline
  \rowcolor{commongray}
  12:30 & \multicolumn{3}{c}{%
    \parbox[c]{24pt}{%
      \includegraphics[height=10pt]{restaurant}%
    }
    Mittagspause
  } \tabularnewline
\end{tabular}

\vspace{0.5\baselineskip}
\noindent\begin{tabular}{Z{0.7cm}Z{2.0cm}Z{2.0cm}Z{2.0cm}}
  & \multicolumn{1}{c}{\cellcolor{geoblau} Hörsaal 1 (0115)}
  & \multicolumn{1}{c}{\cellcolor{hellgelb} Hörsaal 2 (0110)}
  & \multicolumn{1}{c}{\cellcolor{hellgruen} Hörsaal 3 (0119)}
\tabularnewline
  14:00
  \talk{Parkraumanalyse für deine Stadt mit OSM}{Lars Lingner, Alex Seidel, Tobias Jordans}
  \talk{COPC, das neue cloudoptimierte Format für Point Clouds}{Pirmin Kalberer}
  \talk{Ein Frontend für die Legacy-Netzwerkplanung in der Telekommunikation}{Marc Jansen, Matthias Daues}
  \tabularnewline
  14:30
  \talk{Die etwas andere Fahrradkarte}{Christopher Lorenz}
  \talk{Cloudoptimierte Formate für Kacheln und multidimensionale Rasterdaten}{Marco Hugentobler}
  \talk{Nutzung und Support von QGIS in der IT der SachsenEnergie}{Christoph Jung}
  \tabularnewline
  15:00
  \talk{Radverkehrsatlas~-- OSM für die schnelle Planung von Radinfrastruktur}{Tobias Jordans, Boris Hekele}
  \talk{INSPIRE-Metadatensuche in QGIS}{Armin Retterath}
  \talk{Ein Meer an Möglichkeiten~-- Szenariobau für das Offshore-Stromnetz in QGIS}{Felix Jakob Fliegner}
  \tabularnewline
  \rowcolor{commongray}
  15:30 & \multicolumn{3}{c}{%
    \parbox[c]{24pt}{%
      \includegraphics[height=10pt]{cafe}%
    }
    Kaffeepause + Gruppenfoto am Haupteingang ESZ}
    \tabularnewline
\end{tabular}

%\vspace{0.5\baselineskip}
\enlargethispage{1.0\baselineskip}
\renewcommand{\arraystretch}{1.4}
\noindent\begin{tabular}{Z{0.7cm}Z{2.0cm}Z{2.0cm}Z{2.0cm}}
  & \multicolumn{1}{c}{\cellcolor{geoblau} Hörsaal 1 (0115)}
  & \multicolumn{1}{c}{\cellcolor{hellgelb} Hörsaal 2 (0110)}
  & \multicolumn{1}{c}{\cellcolor{hellgruen} Hörsaal 3 (0119)}
  \tabularnewline
  16:00
  \talk{Open Data zu wasserbezogenen Klimarisiken: Wo steht Berlin-Brandenburg?}{Fabio Brill, Tobia Lakes}
  \talk{Geodatenanalyse in der Cloud mit OGC API Processes und pygeoapi}{Jakob Miksch, Hannes Blitza}
  \talk{OSM-Daten und Indoor-Karten in KDE Itinerary}{Volker Krause}
  \tabularnewline
  16:30
  \talk{Open Data, Open Source, Open Berlin}{Lisa Stubert}
  \talk{EO-Lab: SHOGun WebGIS, actinia Rasterprozessierung in der Cloud}{Arnulf B. Christl}
  \talk{Floor plan extraction from digital building models}{Helga Tauscher, Subhashini Krishnakumar}
  \tabularnewline
  17:00
  \talk{FOSS-GIS in der Berliner Verwaltung~-- Ein Erfolgsmodell?}{Enrico Stein, Matthias Schroeder, Manuel Gehlhoff}
  \talk{Geodatenverarbeitung mit Workflow-Engines}{Pirmin Kalberer}
  \talk{Lightning Talks}{}
  \tabularnewline
  \end{tabular}
  \noindent\begin{tabular}{Z{0.7cm}Z{6.85cm}}
  & \multicolumn{1}{c}{\cellcolor{geoblau} Hörsaal 1 (0115)}
  \tabularnewline
  17:25
  \talk{Spontane Lightning Talks}{}
   \tabularnewline
  17:45
  \talk{Der schnelle Weg in die digitale Souveränität~--\linebreak öffentliche Ausschreibungen mit FOSS}{Torsten Friebe, Torsten Wiebke}
   \tabularnewline
  \rowcolor{commongray}
  19:00
  \talk{Mitgliederversammlung FOSSGIS e.\,V.}{}
   \tabularnewline
\end{tabular}\label{endevortraegedonnerstag}

\subsection*{Anwendertreffen und Demo-Sessions am Donnerstag\linebreak\small{\normalfont Vortragsübersicht auf Seite\,\pageref{donnerstag}--\pageref{endevortraegedonnerstag}}}\label{demodonnerstag}

\noindent\begin{tabular}{Z{0.7cm}Z{6.85cm}}
  & \multicolumn{1}{c}{\cellcolor{dezentrot} HS 4 | Demo (0313)}
  \tabularnewline
09:00
\talk{How-To: OSM-Datenqualität mit dem Ohsome Quality Analyst berechnen\linebreak\emph{(45 Min.)}}{Benjamin Herfort}
  \tabularnewline
10:00
  \talk{Wo liegt der Way?}{Roland Olbricht}
  \tabularnewline
  \rowcolor{commongray}
 10:30 & \multicolumn{1}{c}{%
    \parbox[c]{24pt}{%
      \includegraphics[height=10pt]{cafe}%
    }
    Frühstückspause
  } \tabularnewline
  \end{tabular}
  \noindent\begin{tabular}{Z{0.7cm}Z{2.0cm}Z{2.0cm}Z{2.0cm}}
  & \multicolumn{1}{c}{\cellcolor{dezentrot} HS 4 | Demo (0313)}
  & \multicolumn{1}{c}{\cellcolor{audimax} Exp | Anw (1306)}
  & \multicolumn{1}{c}{\cellcolor{eins} Anw (1307)}
  \tabularnewline
  11:00
  \talk{QWC2 und qwc-services\linebreak \emph{(45 Min.)}}{Horst Düster}
  \talk{Zentrale Vermittlung und Technik für die Kooperation von OSM und Naturschutz\newline \emph{(45 Min.)}}{Sebastian Sarx}
  \talk{Neues aus dem OSGeo-Projekt deegree --\linebreak Update 2023\linebreak \emph{(45 Min.)}}{Torsten Friebe}
  \tabularnewline
   12:00
  \talk{Lightning Talks}{}
  \talk{}{}
  \talk{}{}
  \tabularnewline
      \rowcolor{commongray}
    12:30 &
    \multicolumn{3}{c}{%
      \parbox[c]{24pt}{%
        \includegraphics[height=10pt]{restaurant}%
      }
      Mittagspause
    }
    \tabularnewline
    \end{tabular}
    \newpage
   \noindent\begin{tabular}{Z{0.7cm}Z{2.0cm}Z{2.0cm}Z{2.0cm}}
  & \multicolumn{1}{c}{\cellcolor{dezentrot} HS 4 | Demo (0313)}
  & \multicolumn{1}{c}{\cellcolor{audimax} Exp | Anw (1306)}
  & \multicolumn{1}{c}{\cellcolor{eins} Anw (1307)}
  \tabularnewline  
       14:00
       \talk{Visualisierung und Analyse von Satellitenbildern mit der EnMAP-Box\linebreak \emph{(45 Min.)}}{Benjamin Jakimow, Andreas Janz, Fabian Thiel, Patrick Hostert, Sebastian van der Linden}
       \talk{Indoor OSM\linebreak \emph{(60 Min.)}}{Tobias Knerr}
       \talk{Anwendertreffen GBD WebSuite\linebreak \emph{(60 Min.)}}{Otto Dassau}
  \tabularnewline
       15:00
  \talk{Analysis Ready Fernerkundungsdaten erzeugen mit FORCE}{Benjamin Jakimow, Patrick Hostert, David Frantz, Stefan Ernst}
  \talk{}{}
  \talk{}{}
  \tabularnewline
      \rowcolor{commongray}
    15:30 &
    \multicolumn{3}{c}{%
      \parbox[c]{24pt}{%
        \includegraphics[height=10pt]{cafe}%
      }
      Kaffeepause
    }
    \tabularnewline
    16:00
    \talk{ALKIS-NAS-Daten in QGIS und im WebGIS (QGIS-Server mit Lizmap) nutzen\linebreak \emph{(45 Min.)}}{Günter Wagner}
    \talk{Ask me anything QGIS!\linebreak \emph{(60 Min.)}}{Matthias Kuhn, Marco Bernasocchi}
    \talk{GeoNode-Anwendertreffen\linebreak \emph{(60 Min.)}}{Florian Hoedt}
  \tabularnewline
%\end{center}
\renewcommand{\arraystretch}{1.0}
\normalsize\end{tabular}
\newpage
