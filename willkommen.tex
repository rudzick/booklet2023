\newpage
\section*{Willkommen zur FOSSGIS-Konferenz 2023 in Berlin!}\label{welcome}
Die Abkürzung { FOSSGIS} steht für {\bfseries f}reie und {\bfseries O}pen"={\bfseries S}ource"={\bfseries S}oftware für {\bfseries G}eo{\bfseries i}nformations{\bfseries s}ysteme.
Die FOSSGIS-Konferenz 2023 wird vom gemeinnützigen FOSSGIS e.V., der
OpenStreetMap"=Community und der Humboldt-Universität zu Berlin
veranstaltet.
Ziel der jährlich stattfindenden Konferenz ist die Verbreitung von freier,
quelloffener Software für Geoinformationssysteme. In den nächsten vier Tagen
haben Sie die Gelegenheit, sich mit Entwickler:innen und anderen Anwender:innen
auszutauschen und \mbox{neueste} Informationen zu Anwendungen und
Arbeitsmöglichkeiten zu erhalten.

\section*{Anwender- und Expert:innentreffen, Birds of a Feather}
Die FOSSGIS-Konferenz ist eine Communityveranstaltung.
Die gesamte Konferenz über stehen zwei Räume für Anwender- und Expert:innentreffen oder spontan organisierte
Treffen Gleichgesinnter (Birds of a Feather), u.\,ä.
zur Verfügung. Eigene Sessions können Sie an der Pinnwand im Foyer selbst eintragen.
\pagebreak

\section*{Abendveranstaltung am Mittwoch}\label{schwaetzli}
Traditionell findet am ersten Abend der FOSSGIS-Konferenz die
Abendveranstaltung statt, auch Social Event genannt. Der Eintritt
ist im FOSSGIS-Konferenz-Ticket enthalten. Eine Anmeldung ist erforderlich.
Die Abendveranstaltung wird am Konferenzstandort im Erwin-Schrödinger-Zentrum
ab {\bfseries 18:30 bis 22:00 Uhr} stattfinden.

\section*{Mapathon mit Ärzte ohne Grenzen am Freitag und Samstag}
Ärzte ohne Grenzen lädt im Rahmen der FOSSGIS Konferenz 2023 zu einem Mapathon ein. Die Initiative hat das Ziel die Erstellung und Vervollständigung von Kartenmaterial zu unterstützen, welches die Hilfe für Menschen in Krisengebieten erleichtern kann. Es wird eine Einführung geben. Weitere Informationen sind hier zu finden. FOSSGIS-Teilnehmende sind eingeladen, mitzumachen, auch wenn sie noch nie gemappt haben.
Termine: {\bfseries Freitag, 17.00 bis 20.00 Uhr} und {\bfseries Samstag, 11.00 bis 14.00 Uhr}. Bitte melden Sie sich über den Anmeldelink auf der Konferenzwebseite an.
\pagebreak

\section*{Exkursionen}
\subsection*{Kartenabteilung der Staatsbibliothek zu Berlin}
Von den Sektionskarten von Balbi und der Preußischen Uraufnahme zu digitalen Kartendaten.
Am {\bfseries Freitagnachmittag} und {\bfseries Samstagvormittag} werden Führungen durch die Karten\-abteilung der Staatsbibliothek zu Berlin im Haus Unter den Linden angeboten. Anmeldung erforderlich.

\subsection*{Campus Adlershof/WISTA-Gelände}
Die FOSSGIS 2023 findet auf dem Gelände des Wissenschafts- und Wirtschaftsstandort Adlershof (WISTA) statt. Dort existieren noch mehrere technische Denkmale, wie beispielsweise der Trudelturm, die an den Beginn der Luftfahrtindustrie auf dem Campus erinnern. Zu sehen sind aber auch moderne Forschungseinrichtungen, Gründer- und Technologiezentren, die auf die heutige Entwicklung als Zukunftsort für Forschung und Innovation verweisen. In einer kleinen, etwa 60-minütigen Führung können wir sowohl dieser Geschichte als auch den neuen Trends nachstöbern und mehr über den Standort Adlershof erfahren.
Die Exkursion startet am {\bfseries Freitag} um {\bfseries 16:30 Uhr} (nach dem Sektempfang) vor dem Haupteingang des Erwin-Schrödinger-Zentrums. Anmeldung erforderlich.

\section*{Rahmenprogramm am Donnerstag}
\subsection*{Gruppenfoto}
Auch in diesem Jahr wollen wir uns das Gruppenfoto nicht entgehen lassen und laden Sie ein am {\bfseries Donnerstag} in der {\bfseries Nachmittagspause} zum gemeinsamen Gruppenfoto am Haupteingang des Erwin-Schrödinger-Zentrums ein.

\subsection*{Treffen der Freiberufler}
Am {\bfseries Donnerstag} werden sich um {\bfseries 17:45 Uhr} die Freiberufler aus dem FOSSGIS-Bereich im Seminarraum 1\textquotesingle 306 zum gemeinsamen Austausch treffen.

\subsection*{Mitgliederversammlung des FOSSGIS e.\,V.}
Am {\bfseries Donnerstag} sind alle Mitglieder und Gäste ab {\bfseries 18.00 Uhr} herzlich eingeladen, an der Mitgliederversammlung teilzunehmen und sich zu beteiligen. Der FOSSGIS e.\,V. lädt zum Diskutieren, Kennenlernen, Abstimmen und zu Neuwahlen ein. Es wird Getränke und Pizza für alle geben. Der Verein freut sich über zahlreiches Erscheinen.

\section*{Rahmenprogramm am Freitag}
\subsection*{Jeopardy-Quiz}
Am {\bfseries Freitag} gibt es vor dem Abschluss des reguläres Konferenzprogramms um {\bfseries 14.30 Uhr} das nunmehr legendäre und sehr humorvolle Quiz in Form des FOSSGIS-Jeopardy mit Hannes und Tobia.

\subsection*{Sektempfang am FOSSGIS-Stand}
Alle Mitglieder des FOSSGIS-Vereins, Freunde und Interessierte sind am {\bfseries Freitag} ab {\bfseries 16.00 Uhr} herzlich zum Sektempfang zum Ausklang der FOSSGIS 2023 am FOSSGIS-Vereins-Stand eingeladen.

\subsection*{OSM-Event am Freitagabend}
Für alle, die am {\bfseries Freitagabend} noch in der Stadt sind und/oder am OSM-Event teilnehmen möchten gibt es um {\bfseries 19/19:30 Uhr} ein gemeinsames Treffen im Restaurant Greek Food Olympia, Rudower Chaussee 5a in 12489 Berlin-Adlershof. Jeder zahlt seine Rechnung selbst.

\small
\newpage
\label{platinsposoren}
\section*{Platinsponsor und Aussteller}
%\begin{center}
  \includegraphics[width=0.6\textwidth]{001_camptocamp_logo.png}
  %\end{center}
  \vspace{1.0\baselineskip}
  
\noindent
{\bfseries CamptoCamp} gehört zu den führenden Dienstleistern im Bereich Open-Source-GIS und ist in vielen unterschiedlichen Open-Source-Communitys stark engagiert.

\noindent
Unsere Dienstleistungen stützen sich auf 20 Jahre Erfahrung in der Umsetzung von innovativen GIS-Lösungen für Behörden und Unternehmen und erlauben einen hochwertigen und individuellen Service. Das Besondere an Camptocamp sind die hochqualifizierten Mitarbeiter und ihr großes Engagement im "`Ökosystem"' der eingesetzten Open-Source-Software-Lösungen, indem sehr enge Beziehungen zu den Herstellern der jeweiligen Produkte gepflegt werden.

\noindent
Um die oft anspruchsvollen Projekte umzusetzen, erstellt Camptocamp individuelle Lösungen, die auf den am besten geeigneten und fortschrittlichsten Open-Source-Technologien basieren. Camptcamp ist in München, Lausanne, Olten, Paris und Chambéry vertreten und bietet neben Lösungen im GIS-Bereich auch eine große Expertise im den Bereichen ERP (Enterprise-Resource-Planning) und IT-Infrastruktur-Lösungen.

\newpage
\section*{Platinsponsor und Aussteller}
\begin{center}
  \centerline{\includegraphics[width=0.5\textwidth]{002_WhereGroup.jpg}}
\end{center}
\vspace*{-0.4cm}

\noindent
Die {\bfseries WhereGroup GmbH} ist eines der größten Open-Source-Software\-häuser der GEO-IT-Branche in Deutschland. Wir managen zahlreiche komplexe GIS-Projekte unterschiedlichster Art. Als mittelständisches Unternehmen mit über 40 Mitarbeiter*innen an vier Standorten arbeiten wir innovativ und haben dennoch die Bodenhaftung nicht verloren.
Wir bieten Ihnen kompetente Unterstützung in den Bereichen Geographische Informationssysteme (GIS), Web-GIS, Datenbanken, Standards, Interoperabilität und System-Integration. Angefangen bei der Beratung, Konzeption und Entwicklung bis hin zum Betrieb dynamischer Kartenanwendungen im Intra- und Internet.
Grundlage unseres Schaffens ist der Open-Source-Gedanke. Als Teil einer starken Community, mit der wir in engem Austausch stehen, engagieren wir uns aus Überzeugung im FOSSGIS e.\,V., im QGIS-DE e.\,V. und bei der OSGeo. Wir sind aus voller Überzeugung auf Open-Source-Entwicklungen spezialisiert und integrieren professionelle freie Software nahtlos in proprietäre Systeme.
Wir implementieren Lösungen für den Mittelstand, die Industrie und auf allen Ebenen der öffentlichen Verwaltung, beraten und begleiten bei Planung, Migration und Einführung von raumbezogenen Informationssystemen.
Dabei reicht das Spektrum unserer Projekte von Desktop-Lösungen über Geoportalen und kartenbasierter Datenverwaltung bis hin zu hochverfügbaren Anwendungen für die freie Wirtschaft und die öffentliche Verwaltung.
Unser Schulungsinstitut, die FOSS Academy, bietet außerdem praxisorientierte Schulungen zum Thema "`GIS mit Open-Source-Software"' an.
Mehr zur WhereGroup unter www.wheregroup.com und www.foss-academy.com.


\newpage
\section*{Platinsponsor}
%\begin{center}
\begin{flushright}
  \includegraphics[width=0.6\textwidth]{003_hu_siegel-kombi_rgb.png}
 %\end{center}
\end{flushright}
\noindent
Die {\bfseries Humboldt-Universität zu Berlin}, gegründet 1810, ist die älteste Hochschule in Berlin
und eine der renommiertesten Universitäten weltweit. Das Lehr- und Forschungsangebot der HU umfasst heute alle grundlegenden
Wissenschaftsdisziplinen der Geistes-, Sozial- und Kulturwissenschaften, der Rechtswissenschaften, der Lebenswissenschaften, der Mathematik und Naturwissenschaften, der Medizin, der Agrarwissenschaften und der Nachhaltigkeits- und Antikeforschung. Aktuell studieren an der Humboldt-Universität fast 37\,000 junge Menschen aus über 100 Ländern in 171 Bachelor- und Masterstudiengängen betreut von über 400 Professor:innen. Zirka 34 Prozent der wissenschaftlichen Mitarbeiter:innen kommen aus anderen Ländern. Aufgrund zahlreicher Projekte der Spitzenforschung und renommierter internationaler Netzwerke ist die Humboldt-Universität eine der bedeutendsten Universitäten im deutschsprachigen Raum. Bei der Exzellenzstrategie 2019 wurde sie gemeinsam mit den Partner:innen der Berlin University Alliance als Exzellenzverbund ausgezeichnet. Zuvor gehörte sie seit 2012 zu einer der elf deutschen Exzellenzuniversitäten. Die HU verbindet Forschungsexzellenz mit innovativer Nachwuchsförderung. Im Fokus der Lehre stehen forschendes Lernen, Interdisziplinarität und Internationalisierung.

\newpage
\section*{Platinsponsor und Aussteller}
\begin{center}
  \includegraphics[width=0.7\textwidth]{004_TSB_quer.png}
\end{center}
\noindent
Die {\bfseries Technologiestiftung Berlin} ist eine unabhängige und gemeinnützige Stiftung. Wir arbeiten für ein lebenswertes, smartes Berlin~-- und eine lebendige, transparente Stadtgesellschaft, die alle am digitalen Wandel teilhaben lässt. Mit digitalen Tools und smarten Lösungen tragen wir aktiv dazu bei, dass Berlin offen, nachhaltig und effizient wird. Viele unserer Projekte sind Leuchttürme, die beispielhaft die Chancen der Digitalisierung zeigen und Berlin über die Stadtgrenzen hinaus profilieren.

\noindent
Open Data und Open Source gehören zu unserer DNA. Gemeinsam mit Stadtgesellschaft, Verwaltung, Wissenschaft und Unternehmen nutzen wir das Potenzial offener Daten und Anwendungen, um Transparenz zu schaffen, Teilhabe zu ermöglichen und innovative Lösungen zu fördern. Und das in ganz unterschiedlichen Bereichen: Mit der Senatsverwaltung für Inneres, Digitalisierung und Sport bieten wir die \emph{Open Data Informationsstelle} an. Sie unterstützt die Berliner Verwaltung bei der Bereitstellung offener Daten und entwickelt darauf basierend eigene Anwendungen wie die Visualisierung der Berliner Haushaltsdaten oder das Organigramm-Tool. Im Projekt kulturdaten.berlin schaffen wir die erste offene, digitale Infrastruktur für Kulturschaffende und -institutionen in Berlin, gefördert durch die Senatsverwaltung für Kultur und Europa. Und die Open-Map-Anwendung \emph{Gieß den Kiez} vom CityLAB Berlin ermöglicht Bürger:innen die Pflege von über 800\,000 Stadtbäumen.

\noindent
Auf der FOSSGIS freuen wir uns auf den gemeinsamen Austausch und neue Impulse zu offenen Daten sowie Datenbank- und OpenStreetMap-Anwendungen!


\newpage
\section*{Goldponsor}
\begin{flushright}
\includegraphics[width=0.7\textwidth]{101_BKG_Logo_RGB.png}
\end{flushright}
\noindent
Das {\bfseries Bundesamt für Kartographie und Geodäsie (BKG)} ist eine Behörde im Geschäftsbereich des Bundesministeriums des Innern und für Heimat (BMI). Es fungiert als zentraler Dienstleister des Bundes und Kompetenzzentrum für Geoinformation und geodätische Referenzsysteme. Das BKG befasst sich mit der Beobachtung sowie der Datenhaltung bis hin zur Analyse, Kombination und Bereitstellung von Geodaten. Das BKG ermöglicht aufgrund der zentralen Geodatenbereitstellung eine optimale und wirtschaftliche Geodatennutzung im Bundesbereich.

\noindent
Das BKG setzt sich für eine offene Datenpolitik ein, wodurch die Verbreitung von Open Data gefördert wird. Dies schließt die Beratung anderer Bundesbehörden beim Umgang mit OSM-Daten ein. Die Nutzung, Entwicklung und Verbreitung der Nutzung freier Software liegen ebenfalls im Bereich der Aktivitäten.

\noindent
Von der Arbeit des BKG profitieren insbesondere Bundeseinrichtungen, die öffentliche Verwaltung, Wirtschaft, Wissenschaft~-- und fast jeder Bürger in Deutschland. Experten aus den verschiedensten Bereichen wie Verkehr, Katastrophenvorsorge, Innere Sicherheit, Energie und Umwelt verwenden Geodaten, Landkarten, Referenzsysteme und Informationsdienste des BKG für ihre Pläne und Untersuchungen. Das BKG unterhält ein Dienstleistungszentrum in Leipzig sowie geodätische Observatorien im In- und Ausland.

\newpage
\section*{Goldponsor und Aussteller}
\begin{flushright}
  \includegraphics[width=0.8\textwidth]{102_Logo_QFieldCloud-by-OpenGIS_transparent.png}
\end{flushright}
\noindent
    {\bfseries OPENGIS.ch GmbH} plant und entwickelt personalisierte Open-Source-GIS"=Lösungen als Desktop-, Web- oder Mobilapplikationen für Ingenieurbüros, Organisationen und den öffentlichen Sektor~-- kosteneffizient, massgeschneidert und von A bis Z. Mit Open-Source-Technologie-Erfahr\-ung, POSTGIS-Expertise und QGIS- und QField"=Entwicklerwissen finden wir elegante Lösungen auch für komplexe Aufgaben. Darauf geben wir Ihnen unser schweizer GeoNinja-Ehrenwort~-- oder einen Supportvertrag mit SLA.
    
\noindent
Aber nicht nur wir sind von uns überzeugt. Viel wichtiger, auch die Deutsche Bahn, das Bundesamt für Umwelt, mehrere Kantone und weitere Auftraggeber sind sich einig: OPENGIS.ch ist der ideale Partner, wenn es um Open-Source-GIS-Projekte geht.

\noindent
QFieldCloud ergänzt die mobile Applikation QField für die Synchronisierung der erfassten Daten und erleichtert die Zusammenarbeit von mehreren Personen oder Teams im Feld. Nutzer und Rollen werden klar definiert, Änderungen können nachverfolgt und erfasste Daten einfach über die Cloud synchronisiert werden.

\normalsize

