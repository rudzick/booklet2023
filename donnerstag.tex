\enlargethispage{2.0\baselineskip}
% time: Thursday 09:00
% URL: https://pretalx.com/fossgis2023/talk/fossgis2023-23771-20-jahre-qgis/

%
\newTimeslot{09:00}
\noindent\abstractHSeins{%
  Marco Bernasocchi%
}{%
  20 Jahre QGIS%
}{%
}{%
  QGIS feierte seinen 20 Geburtstag letztes Jahr. In diesem Vortrag werde ich auf die wichtigsten
  Features und Events dieser Jahre eingehen, welche QGIS mit seiner Community zu einem der 10
  wichtigsten C++-Open-Source-Projekte machte.
  Und allgemein ein fantastisches Projekt zum Repräsentieren.%
}%


%%%%%%%%%%%%%%%%%%%%%%%%%%%%%%%%%%%%%%%%%%%

% time: Thursday 09:00
% URL: https://pretalx.com/fossgis2023/talk/fossgis2023-23425-xplan-reader-ein-qgis-plugin/

%

\noindent\abstractHSzwei{%
  Michael Stein%
}{%
  XPlan-Reader~-- ein QGIS-Plugin%
}{%
}{%
  XPlanung ist der gesetzlich verbindliche Standard für B-Plan, FNP, LRP. 
Am Ende jedes Planverfahrens steht neben der Urkunde dann eine GML-Datei, für
  die jedoch kein Open-Source-Vierwer vorgesehen ist. Das QGIS-Plugin \emph{XPLAN Reader} schließt diese
  Lücke.%
}%


%%%%%%%%%%%%%%%%%%%%%%%%%%%%%%%%%%%%%%%%%%%

% time: Thursday 09:00
% URL: https://pretalx.com/fossgis2023/talk/fossgis2023-23477-neues-von-actinia/

%

\noindent\abstractHSdrei{%
  Carmen Tawalika%
}{%
  Neues von actinia%
}{%
}{%
  Hallo nochmal, mein Name ist actinia. Ich bin eine REST-API für GRASS GIS, die die Verwaltung und
  Visualisierung von Locations, Mapsets und Geodaten sowie die Ausführung von GRASS-GIS-Modulen
  ermöglicht und ich kann in einer Cloud-Umgebung installiert werden. Neben diesen Grundlagen gibt
  es auch eine Menge über das vergangene Jahr zu erzählen! Stichpunkte sind
  Client-Implementierungen, Worker/Queues, Tiling- und Parallel-Plugin, STAC und openEO. Um mehr
  Details zu erfahren, kommen Sie vorbei!%
}%


%%%%%%%%%%%%%%%%%%%%%%%%%%%%%%%%%%%%%%%%%%%

% time: Thursday 09:00
% URL: https://pretalx.com/fossgis2023/talk/fossgis2023-23883-how-to-osm-datenqualitt-mit-dem-ohsome-quality-analyst-berechnen/

%

\noindent\abstractHSvier{%
  Benjamin Herfort%
}{%
  How-To: OSM Datenqualität mit dem Ohsome Quality Analyst berechnen%
}{%
}{%
  Wir wollen die Software \emph{Ohsome Quality Analyst} vorführen. In der Demo-Session zeigen wir wie
  jeder in wenigen Schritten für ausgewählte Anwendungsfälle und Regionen verschiedene
  Qualitätsaspekte von OSM (Vollständigkeit, Aktualität, Detailliertheit,~\ldots) berechnen kann. Wir
  möchten die Demo auch dazu nutzen zu diskutieren, wie wir unterschiedliche Auffassungen von
  Qualität berücksichtigen können und welche Aspekte für die deutsche OSM Community von besonderer
  Relevanz sind.%
}%
\vspace{1.92\baselineskip}
\small
\sponsorBoxA{404_52North.png}{0.35\textwidth}{4}{%
\textbf{Bronzesponsor und Aussteller}
\vspace{0.5\baselineskip}

\noindent {\bfseries 52°North Spatial Information Research GmbH} ist ein angewandtes Forschungsunternehmen im Bereich der Geo-Dateninfrastrukturen und der Ableitung von räumlichen Informationsprodukten. Als Non-Profit-Unternehmen unterstützt 52°North Open Science durch offene Daten und Open-Source-Software. Unser Hauptinteresse gilt der Entwicklung räumlicher Forschungsdateninfrastrukturen, um die Ableitung von Informationen aus Daten zu erleichtern und zu fördern.
}%
\normalsize
%

%%%%%%%%%%%%%%%%%%%%%%%%%%%%%%%%%%%%%%%%%%%

% time: Thursday 09:30
% URL: https://pretalx.com/fossgis2023/talk/fossgis2023-23473-seekarten-mit-qgis-geht-das-/

%
\newTimeslot{09:30}
\noindent\abstractHSeins{%
  Ulrike Assmann, Annette Hey%
}{%
  Seekarten mit QGIS~-- geht das?%
}{%
}{%
  Das Bundesamt für Seeschifffahrt und Hydrographie erstellt amtliche Seekarten mit einer
  proprietären Spezialsoftware. Aktuell wird geprüft, ob Alternativen nutzbar sind. Dabei wird in
  Richtung digitale Souveränität und noch umfassendere Automatisierung gedacht. In einer ersten
  Stufe wurde eine Machbarkeitsstudie erstellt, welche die Möglichkeiten von QGIS zur
  automatisierten Erstellung von Seekarten testen sollte. Der Vortrag stellt die besonderen
  Schwierigkeiten und erste Ergebnisse vor.%
}%


%%%%%%%%%%%%%%%%%%%%%%%%%%%%%%%%%%%%%%%%%%%

% time: Thursday 09:30
% URL: https://pretalx.com/fossgis2023/talk/fossgis2023-23865-xplanung-mit-open-source-software/

%

\noindent\abstractHSzwei{%
  Torsten Friebe%
}{%
  XPlanung mit Open-Source-Software%
}{%
}{%
  Im April 2022 wurde der Quellcode der Software xPlanBox der Firma lat/lon im Rahmen eines
  Pilotprojekts auf der OpenCoDE-Plattform des BMI veröffentlicht. Die Software liegt nun als
  Open-Source-Lösung vor und kommt im Rahmen des Onlinezugangsgesetz (OZG) und des
  "`Einer-für-Alle"-Prinzips (EfA) zum Einsatz. Der Vortrag stellt kurz die OpenCoDE-Plattform sowie
  deren Funktionen vor und zeigt dann welche Komponenten Teil der Open-Source-Lösung sind.%
}%


%%%%%%%%%%%%%%%%%%%%%%%%%%%%%%%%%%%%%%%%%%%

% time: Thursday 09:30
% URL: https://pretalx.com/fossgis2023/talk/fossgis2023-23591-geoserver-cloud-ein-projektupdate-und-erfahrungsberichte-aus-produktiven-umgebungen/

%

\noindent\abstractHSdrei{%
  Andrea Borghi%
}{%
  GeoServer-Cloud~-- ein \mbox{Projektupdate} und Erfahrungsberichte aus\linebreak produktiven Umgebungen%
}{%
}{%
  Der für Cloud-Umgebungen konzipierte GeoServer-Cloud ist nun bereit für den Einsatz in produktiven
  Serverumgebungen. Dieser Vortrag umfasst ein Update zum Stand dieses Open-Source Projekts und
  erste Erfahrungsberichte aus laufenden Projekten.%
}%
\vspace{1.92\baselineskip}
\small
\sponsorBoxA{405_opencage-logo.png}{0.48\textwidth}{6}{%
\textbf{Bronzesponsor}\\
\noindent {\bfseries OpenCage GmbH} aus Hannover bietet eine auf offenen Daten\linebreak basierende Geokodierung API (forward und reverse geocoding). Sei es einmalige Datenverarbeitung, Systemintegration oder maßgeschneiderte highly-available Systeme für Millionen Anfragen am Tag. Wir organisieren Geomob Tech Talks und Podcast.
}%
\normalsize
%


%%%%%%%%%%%%%%%%%%%%%%%%%%%%%%%%%%%%%%%%%%%

% time: Thursday 10:00
% URL: https://pretalx.com/fossgis2023/talk/fossgis2023-23860-qfieldcloud-effiziente-zusammenarbeit-im-feld/



%
\newTimeslot{10:00}
\noindent\abstractHSeins{%
  Marco Bernasocchi%
}{%
  QFieldCloud~-- effiziente Zusammen\-arbeit im Feld%
}{%
}{%
  QFieldCloud ergänzt die mobile Applikation QField für die Synchronisierung der erfassten Daten und
  erleichtert die Zusammenarbeit von mehreren Personen oder Teams im Feld. Nutzer und Rollen werden
  klar definiert, Änderungen können nachverfolgt und erfasste Daten einfach über die Cloud
  synchronisiert werden.%
}%


%%%%%%%%%%%%%%%%%%%%%%%%%%%%%%%%%%%%%%%%%%%

% time: Thursday 10:00
% URL: https://pretalx.com/fossgis2023/talk/fossgis2023-23712-kommonitor-kommunales-monitoring-zur-raumentwicklung/

%

\noindent\abstractHSzwei{%
  Sebastian Drost, Christian Danowski-Buhren%
}{%
  KomMonitor~-- kommunales Monitoring zur Raumentwicklung%
}{%
}{%
  Die Open-Source-Software KomMonitor ermöglicht ein GIS-basiertes, raum-zeitliches Monitoring von
  Geodaten und Statistiken. Die Software wurde für Kommunalverwaltungen als webbasiertes
  Planungswerkzeug entwickelt, um aktuelle Fragen der Stadtentwicklung beantworten zu können.%
}%

%%%%%%%%%%%%%%%%%%%%%%%%%%%%%%%%%%%%%%%%%%%

% time: Thursday 10:00
% URL: https://pretalx.com/fossgis2023/talk/fossgis2023-23527-neues-aus-dem-geonode-projekt/

%

\noindent\abstractHSdrei{%
  Florian Hoedt%
}{%
  Neues aus dem GeoNode-Projekt%
}{%
}{%
  Was passiert, wenn sich mehrere große Forschungseinrichtungen auf eine Plattform für das
  Forschungsdatenmanagement einigen und gemeinsam neue Features entwickeln (lassen)?%
}%


%%%%%%%%%%%%%%%%%%%%%%%%%%%%%%%%%%%%%%%%%%%

% time: Thursday 10:00
% URL: https://pretalx.com/fossgis2023/talk/fossgis2023-23813-wo-liegt-der-way-/

%

\noindent\abstractHSvier{%
  Roland Olbricht%
}{%
  Wo liegt der Way?%
}{%
}{%
  Das zentrale Datenelement in OpenStreetMap sind Ways, und nützlich sind diese nur mit Koordinaten.
  Das heutige Datenmodell erlaubt es zwar, die Koordinaten eines aktuellen Ways zweifelsfrei aus den
  Koordinaten der referenzierten aktuellen Nodes zu gewinnen, aber bei der Rekonstruktion
  historischer Daten sieht das anders aus. Praktisch relevant ist das für das Sichten von
  Änderungen.%
}%
\vspace{1.92\baselineskip}
\small
\sponsorBoxA{406_Wagner-IT.png}{0.35\textwidth}{5}{%
\textbf{Bronzesponsor}\\
\noindent {\bfseries Wagner-IT} ist seit 2011 für kleinere Gemeinden und Städte im Bereich der GIS- und WebGIS-Betreuung,\linebreak sowie des Geodatenmanagements tätig. Realisiert werden individuelle Lösung auf Basis von QGIS, QGIS-Server, PostgreSQL und WebClients (z.Bsp. dem Lizmap-Client). https://wagner-it.de
}%
\normalsize
%

%%%%%%%%%%%%%%%%%%%%%%%%%%%%%%%%%%%%%%%%%%%

% time: Thursday 11:00
% URL: https://pretalx.com/fossgis2023/talk/fossgis2023-23837-neues-aus-dem-osgeo-projekt-deegree-update-2023/

%
\newTimeslot{11:00}
\noindent\abstractAnwBoF{%
  Torsten Friebe%
}{%
  Neues aus dem OSGeo-Projekt\linebreak deegree~-- Update 2023%
}{%
}{%
  Das OSGeo-Projekt deegree stellt seit 20 Jahren umfassende Referenzimplementierungen für
  OGC-Standards wie WFS, WMS und WMTS sowie seit 2021 auch für OGC API bereit. Der Vortrag geht auf
  die Neuerungen in der aktuellen Version 3.5 und die Unterstützung des neuen Standards OGC API
  Features ein.%
}%


%%%%%%%%%%%%%%%%%%%%%%%%%%%%%%%%%%%%%%%%%%%

% time: Thursday 11:00
% URL: https://pretalx.com/fossgis2023/talk/fossgis2023-23835-zentrale-vermittlung-und-technik-fr-die-kooperation-von-osm-und-naturschutz/

%

\noindent\abstractExpBoFAnw{%
  Sebastian Sarx%
}{%
  Zentrale Vermittlung und Technik für die Kooperation von OSM und Naturschutz%
}{%
}{%
  Die Relevanz der OSM-Daten für die Natur wird immer größer. Konflikte in der digitalen sowie
  realen Welt sind oftmals durch falsche Anwendung von OSM oder fehlendes Wissen geprägt. Diese
  können durch gemeinsame Strategien, Kommunikation und Wissen vermieden werden. Mit einer zentralen
  Stelle für Naturschutz und OSM sowie technischer Lösungen kann eine qualitative Datenpflege im
  Sinne von OSM und dem Naturschutz geschaffen werden. Wir möchten uns austauschen, anregen und zum
  Mitmachen einladen.%
}%
\pagebreak
\enlargethispage{2.0\baselineskip}
%%%%%%%%%%%%%%%%%%%%%%%%%%%%%%%%%%%%%%%%%%%

% time: Thursday 11:00
% URL: https://pretalx.com/fossgis2023/talk/fossgis2023-23456-ablsung-proprietrer-kartografie-software-durch-eine-opensource-und-opendata-lsung/

%

\noindent\abstractHSeins{%
  Robert Klemm, Luise Leffmann%
}{%
  Ablösung proprietärer Kartografie-Software durch eine Open-Source- und Open Data-Lösung%
}{%
}{%
  Im Vortrag wird vorgestellt, wie die bisherige Lösung durch Open-Source-Software (QGIS, PostGIS)
  und Open Data (OpenStreetMap) abgelöst wird. Dabei galt es nicht nur eine technische Lösung zu
  finden, die es ermöglicht, die Kartenbasis aus OSM-Daten zu bestücken und mit den Daten der BVG
  (Liniennetz, Haltestellen etc.) in einer gemeinsamen Datenbank zu kombinieren. Es mussten auch die
  hohen Ansprüche an die kartografische Gestaltung berücksichtigt werden.%
}%


%%%%%%%%%%%%%%%%%%%%%%%%%%%%%%%%%%%%%%%%%%%

% time: Thursday 11:00
% URL: https://pretalx.com/fossgis2023/talk/fossgis2023-23880-schnelle-flexible-volltextsuche-in-openstreetmap/

%

\noindent\abstractHSzwei{%
  Wolfram Schneider%
}{%
  Schnelle, flexible Volltextsuche in OpenStreetMap%
}{%
}{%
  Volltextsuche in mehr als 140 Millionen Namen, Notizen und Beschreibungen von Tags in
  OpenStreetMap nach Wörtern, Teilstrings, Phrasen, regulären Ausdrücken oder bester
  Übereinstimmung.%
}%
\vspace{0.4\baselineskip}
\small
\sponsorBoxA{407_GBD.png}{0.2\textwidth}{2}{%
\textbf{Bronzesponsor und Aussteller}

\noindent Die {\bfseries Geoinformatikbüro Dassau GmbH} aus \mbox{Düsseldorf} bietet seit 2006 Beratung, Konzeption, Schulung, Wartung, Support und Programmierung zum Thema GIS und GDI auf Open Source Basis. Ein Fokus liegt auf der Software QGIS, QGIS Server, QGIS Web Client, GBD WebSuite, Postgres/PostGIS und GRASS GIS.
}%
\normalsize
%


%%%%%%%%%%%%%%%%%%%%%%%%%%%%%%%%%%%%%%%%%%%

% time: Thursday 11:00
% URL: https://pretalx.com/fossgis2023/talk/fossgis2023-23852-qfield-news-navigations-profil-und-ausstecktool-qr-codes-ios-und-vieles-mehr/

%

\noindent\abstractHSdrei{%
  Matthias Kuhn%
}{%
  QField News: Navigations-, Profil- und Ausstecktool, QR-Codes, iOS und vieles mehr%
}{%
}{%
  Die wichtigsten zwischen Juli 2022 und März 2023 entwickelten Features für die Feldapplikation
  QField werden vorgestellt. Dazu gehören das Release der iOS-Version, die Möglichkeit QR-Codes
  einzulesen oder neue Aussteck- und Navigationswerkzeuge.%
}%


%%%%%%%%%%%%%%%%%%%%%%%%%%%%%%%%%%%%%%%%%%%

% time: Thursday 11:00
% URL: https://pretalx.com/fossgis2023/talk/fossgis2023-23839-qwc2-und-qwc-services/

%

\noindent\abstractHSvier{%
  Horst Düster%
}{%
  QWC2 und qwc-services%
}{%
}{%
  Einführung in das System QWC2 und die bestehenden Services. Aufbau einer Demoinstanz während der
  Session.%
}%
\vspace{2.52\baselineskip}
\small
\sponsorBoxA{408_logo_latlon_web.png}{0.32\textwidth}{6}{%
\textbf{Bronzesponsor}\\
\noindent {\bfseries lat/lon GmbH} ist ein Geo-IT-Unternehmen mit Schwerpunkt auf Beratung und Software\-entwicklung in den Bereichen Geodaten\-infrastruktur, Standards und Open Source. Als Mitglied im OGC engagieren wir uns bei Referenzimplementierungen und Konformitätstests. lat/lon ist wesentlich für die Entwicklung des OSGeo-Projekts deegree verantwortlich.
}%
\normalsize
%

%%%%%%%%%%%%%%%%%%%%%%%%%%%%%%%%%%%%%%%%%%%

% time: Thursday 11:30
% URL: https://pretalx.com/fossgis2023/talk/fossgis2023-23857-vollstndige-beschriftung-in-einem-qgis-stadtplanprojekt-mit-hilfe-der-maplex-label-e/

%
\newTimeslot{11:30}
\enlargethispage{3\baselineskip}

\noindent\abstractHSeins{%
  Larissa Bitterich%
}{%
  Vollständige Beschriftung in einem QGIS-Stadtplanprojekt mit Hilfe der Maplex-Label-E%
}{%
}{%
  Der Regionalverband Ruhr stellt seinen Mitgliedskommunen ein QGIS-Projekt zur Verfügung, mit dem
  diese mit geringem Aufwand gedruckte Stadtpläne erstellen können (u.a. für den
  Katastrophenschutz). Die vollständige Beschriftung aller Straßen im Maßstab 1:15\,000 ist eine
  wichtige Anforderung, konnte aber mit QGIS-Werkzeugen nicht erfüllt werden. Stattdessen haben wir
  einen Workflow erstellt, der die deutlich besseren Ergebnisse der Maplex-Label-Engine aus der
  ESRI-Welt in QGIS nutzbar macht.%
}%


%%%%%%%%%%%%%%%%%%%%%%%%%%%%%%%%%%%%%%%%%%%

% time: Thursday 11:30
% URL: https://pretalx.com/fossgis2023/talk/fossgis2023-23916-vektortile-erfahrungen-mit-shortbread/

%

\noindent\abstractHSzwei{%
  Michael Reichert%
}{%
  Vektortile-Erfahrungen mit Shortbread%
}{%
}{%
  Das neue Vektortile-Schema Shortbread wird in der Geofabrik zum Rendern mehrere Kartenstile
  eingesetzt. Die Vektortiles werden mit Tilemaker erzeugt und anschließend mit Mapnik als
  Rastertiles gerendert. Mapnik selbst greift mit dem GDAL-MVT-Treiber auf die Vektortiles zu. Das
  Rendering erfolgt mit Tirex.
  Der Vortrag berichtet über die Hürden, die genommen wurden und die Besonderheiten von GDAL, die zu
  gewissen Anforderungen an die Inhalte der Vektortiles führen.%
}%
\pagebreak

%%%%%%%%%%%%%%%%%%%%%%%%%%%%%%%%%%%%%%%%%%%

% time: Thursday 11:30
% URL: https://pretalx.com/fossgis2023/talk/fossgis2023-23701-mobile-erfassung-von-einfahrten-mit-qgis-und-merginmaps/

%

\noindent\abstractHSdrei{%
  Lars Lingner%
}{%
  Mobile Erfassung von Einfahrten mit QGIS und merginmaps%
}{%
}{%
  Daten zu Grundstückseinfahrten können für Parkraumanalysen ein wichtiger Bestandteil sein. Diese
  Daten sind zumindest in Berlin nicht flächendeckend und nicht aktuell verfügbar. Zur Erfassung
  kann QGIS und merginmaps genutzt werden. Der Praxisbericht zeigt ob das geht und welche
  Erfahrungen damit gesammelt wurden.%
}%
\vspace{1.92\baselineskip}
\small
\sponsorBoxA{410_Geoinfo_Logo.png}{0.32\textwidth}{2}{%
\textbf{Bronzesponsor}\\
\noindent {\bfseries Die GEOINFO Applications AG} verbindet die Leidenschaft für Geoinformationen mit der Begeisterung für Softwaretechnologien. Aus den Ideen unserer Kundschaft entstehen innovative Lösungen. Darunter bedürfnisgerechte Fachanwendungen für Infrastruktur, Sicherheit, Vegetation und Landwirtschaft.
}%
\normalsize
%

%%%%%%%%%%%%%%%%%%%%%%%%%%%%%%%%%%%%%%%%%%%

% time: Thursday 12:00
% URL: https://pretalx.com/fossgis2023/talk/fossgis2023-23832-gelnde-kartierpraktikum-zur-erhebung-von-barriereinformationen/

%
\newTimeslot{12:00}
\noindent\abstractHSeins{%
  Christian Willmes%
}{%
  Gelände- und Kartierpraktikum zur Erhebung von Barriereinformationen%
}{%
}{%
  Der Beitrag beschreibt die Entwicklung und Durchführung eines Gelände- und Kartiepraktikum zur
  Erfassung von Barrieren auf dem Campus der Uni Köln mit Hilfe von Open-Source-Mobile-Apps.%
}%


%%%%%%%%%%%%%%%%%%%%%%%%%%%%%%%%%%%%%%%%%%%

% time: Thursday 12:00
% URL: https://pretalx.com/fossgis2023/talk/fossgis2023-23700-generalisierung-von-osm-daten-mit-osm2pgsql/

%

\noindent\abstractHSzwei{%
  Jochen Topf%
}{%
  Generalisierung von OSM-Daten mit osm2pgsql%
}{%
}{%
  Die automatische Generalisierung von Kartendaten ist eine große Herausforderung und es gibt noch
  nicht viel Open-Source-Software in diesem Bereich. In einem Halbjahresprojekt, das vom Prototype
  Fund bzw. dem BMBF gefördert wurde, habe ich osm2pgsql (eine Software zum Import von
  OpenStreetMap-Daten in eine PostGIS-Datenbank) um Elemente zur Generelisierung von
  Kartendaten erweitert. In dem Vortrag will ich die Ergebnisse dieses Projektes vorstellen.%
}%
\pagebreak

%%%%%%%%%%%%%%%%%%%%%%%%%%%%%%%%%%%%%%%%%%%

% time: Thursday 12:00
% URL: https://pretalx.com/fossgis2023/talk/fossgis2023-23829-qgis-und-postgis-nebst-qfield-und-qgis-server-im-einsatz-bei-der-entsorgung-dortmund/

%

\noindent\abstractHSdrei{%
  Jörg Thomsen, Jakob Kopec%
}{%
  QGIS und PostGIS nebst QField und QGIS-Server im Einsatz bei der Entsorgung Dortmund%
}{%
}{%
  Die Entsorgung Dortmund GmbH nutzt seit vielen Jahren QGIS und PostGIS in verschiedenen
  Arbeitsbereichen für die Dokumentation und Planung der vielfältigen täglichen Arbeit. In der
  jüngeren Vergangenheit sind QField und QGIS-Server als weitere Komponenten für den Einsatz vor Ort
  hinzu gekommen.%
}%


%%%%%%%%%%%%%%%%%%%%%%%%%%%%%%%%%%%%%%%%%%%

% time: Thursday 12:00
% URL: https://pretalx.com/fossgis2023/talk/fossgis2023-23888-data-mundialis-eine-sammlung-neuer-offener-datenprodukte-/

%

\noindent\abstractHSvier{%
  Markus Neteler%
}{%
  data.mundialis: eine Sammlung\linebreak neuer offener Datenprodukte!%
}{%
}{%
  Mundialis hat eine Reihe an neuen und offenen Datenprodukten erstellt (Landnutzungsklassifikation,
  Gebäudeextraktion, Zeitreihen kontinentaler Rasterdaten aus ERA5-Land-Daten), die wir in diesem
  Lightning Talk vorstellen werden. Sie sind auf https://zenodo.org und im
  Geonetwork-Metadatakatalog auf https://data.mundialis.de verfügbar.%
}%
\pagebreak

%%%%%%%%%%%%%%%%%%%%%%%%%%%%%%%%%%%%%%%%%%%

% time: Thursday 12:05
% URL: https://pretalx.com/fossgis2023/talk/fossgis2023-23476-legendenbilder-aus-vector-tile-styles-ableiten/

%
\newSmallTimeslot{12:05}
\noindent\abstractHSvier{%
  Sebastian Ratjens%
}{%
  Legendenbilder aus Vectortile-Styles ableiten%
}{%
}{%
  Mit der auf MapLibre basierenden Open-Source-Anwendung \emph{vt-legend} können Legendenbilder anhand
  eines Vectortile-Styles und einer zugehörigen Legendenkonfiguration abgeleitet werden. Der
  Vortrag gibt einen kurzen Einblick in die Funktionalitäten und Konfigurationsmöglichkeiten der
  Anwendung.%
}%

%%%%%%%%%%%%%%%%%%%%%%%%%%%%%%%%%%%%%%%%%%%

% time: Thursday 12:10
% URL: https://pretalx.com/fossgis2023/talk/fossgis2023-23890-20-jahre-mapbender/

%
\newSmallTimeslot{12:10}
\noindent\abstractHSvier{%
  Astrid Emde%
}{%
  20 Jahre Mapbender%
}{%
}{%
  Wie die Zeit vergeht. Mapbender wird 2023 nun 20 Jahre alt. In der Vortrag wollen wir
  zurückblicken auf die Stationen der Software  und des Open-Source-Projektes. Und dabei auch die
  Verbindung zu FOSSGIS und OGC beleuchten.%
}%
\vspace{1.92\baselineskip}
\small
\sponsorBoxA{413_DBG.jpg}{0.37\textwidth}{5}{%
\textbf{Bronzesponsor und Aussteller}\\
\noindent Die {\bfseries d.b.g. Datenbankgesellschaft mbH} bietet Softwarelösungen und Dienstleistungen für ein innovatives Freiraummanagement. Das Angebot reicht von Lösungen für die Erfassung von Daten über die Planung bis zur Betriebssteuerung. Überdies stellt die d.b.g QGIS-Knowhow zur Verfügung.
}%
\normalsize
%

%%%%%%%%%%%%%%%%%%%%%%%%%%%%%%%%%%%%%%%%%%%

% time: Thursday 12:15
% URL: https://pretalx.com/fossgis2023/talk/fossgis2023-23846-die-qgis-anwendergruppen-deutschland-und-schweiz-stellen-sich-vor/

%
\newTimeslot{12:15}
\noindent\abstractHSvier{%
  Isabel Kiefer%
}{%
  Die QGIS-Anwendergruppen Deutschland und Schweiz stellen sich vor%
}{%
}{%
  Das QGIS.org-Projekt hat lokale Anwendergruppen in vielen Ländern. Diese bieten eine Plattform für
  die Verknüpfung und den Austausch von Nutzern, für lokal interessante Weiterentwicklungen oder
  länderspezifische Module. Sie tragen zur Vermarktung von QGIS innerhalb des jeweiligen Landes bei
  und unterstützen QGIS.org oft finanziell durch einen Teil ihrer Mitgliederbeiträge. Die QGIS-Anwendergruppen
  aus Deutschland und der Schweiz stellen ihre Ziele und ihre Arbeitsweise vor.%
}%
\vspace{1.92\baselineskip}
\small
\sponsorBoxA{411_geoSYS_logo.png}{0.35\textwidth}{5}{%
\textbf{Bronzesponsor und Aussteller}\\
\noindent {\bfseries GeoSYS} ist Dienstleister im Bereich Geoinformation. Wir beraten Unternehmen, Verwaltungen bei der Einführung von GIS, Geodatenbanken, Geodateninfrastrukturen und Webmapping-Lösungen. Wir entwickeln Anwendungen, Portale, Geo-Apps und auch GIS-Plugins und setzen Projekte in aller Welt um.
}%
\normalsize
%

%%%%%%%%%%%%%%%%%%%%%%%%%%%%%%%%%%%%%%%%%%%

% time: Thursday 14:00
% URL: https://pretalx.com/fossgis2023/talk/fossgis2023-23703-gbd-websuite-anwendertreffen/

%
\newTimeslot{14:00}
\noindent\abstractAnwBoF{%
  Otto Dassau%
}{%
  Anwendertreffen GBD WebSuite%
}{%
}{%
  Die GBD WebSuite (https://gbd-websuite.de) ist ein Open-Source-Anwendungs- und Webserver mit dem
  Schwerpunkt Geodatenverarbeitung. Sie wird seit 2017 entwickelt und deutschlandweit eingesetzt.
  Sie kann als OCI-Container plattformunabhängig in bestehende IT-Infrastrukturen integriert werden.
  Wir möchten im Rahmen des Anwendertreffens den neuen Release 8 präsentieren und aktuellen Nutzern
  und Interessierten die Möglichkeit bieten, sich kennenzulernen, auszutauschen und Fragen zu
  stellen.%
}%


%%%%%%%%%%%%%%%%%%%%%%%%%%%%%%%%%%%%%%%%%%%

% time: Thursday 14:00
% URL: https://pretalx.com/fossgis2023/talk/fossgis2023-23382-indoor-osm/

%

\noindent\abstractExpBoFAnw{%
  Tobias Knerr%
}{%
  Indoor OSM%
}{%
}{%
  Ein Treffen für alle, die als Mapper oder Entwickler mit Indoor-Karten in OpenStreetMap zu tun
  haben.%
}%


%%%%%%%%%%%%%%%%%%%%%%%%%%%%%%%%%%%%%%%%%%%

% time: Thursday 14:00
% URL: https://pretalx.com/fossgis2023/talk/fossgis2023-23697-parkraumanalyse-fr-deine-stadt-mit-openstreetmap/

%

\noindent\abstractHSeins{%
  Tobias Jordans, Lars Lingner, Alex Seidel%
}{%
  Parkraumanalyse für deine Stadt mit OpenStreetMap%
}{%
}{%
  Seit zwei Jahren arbeiten wir~-- weitgehend ehrenamtlich~-- an einem Prozess um Parkraumanalysen auf
  Basis von OpenStreetMap zu ermöglichen. Jetzt ist das Tooling und die Dokumentation soweit, dass
  auch du für deine Stadt Parkraumdaten auf Basis von OSM erfassen kannst. Wir zeigen, wie wir den
  Prozess automatisiert haben und welche Erfahrungen wir gesammelt haben.%
}%

\newpage
%%%%%%%%%%%%%%%%%%%%%%%%%%%%%%%%%%%%%%%%%%%
\enlargethispage{3\baselineskip}

% time: Thursday 14:00
% URL: https://pretalx.com/fossgis2023/talk/fossgis2023-23867-copc-das-neue-cloudoptimierte-format-fr-point-clouds/

%

\noindent\abstractHSzwei{%
  Pirmin Kalberer%
}{%
  COPC, das neue cloudoptimierte Format für Point-Clouds%
}{%
}{%
  Neben dem bereits etablierten Format COG für Cloud-optimized GeoTIFFs finden cloudoptimierte
  Formate auch für andere Anwendungen immer mehr Interesse. Für Point-Cloud-Daten existiert das COPC-Format,
  welches auf dem bewährten offenen LASzip-Format für verlustfrei
  komprimierte LIDAR-Daten aufbaut. Wie COGs sind die Daten rückwärtskompatibel, was die
  Implementation stark vereinfacht und die Adaption fördert.%
}%


%%%%%%%%%%%%%%%%%%%%%%%%%%%%%%%%%%%%%%%%%%%

% time: Thursday 14:00
% URL: https://pretalx.com/fossgis2023/talk/fossgis2023-23872-ein-frontend-fr-die-legacy-netzwerkplanung-in-der-telekommunikation/

%

\noindent\abstractHSdrei{%
  Matthias Daues, Marc Jansen%
}{%
  Ein Frontend für die Legacy-Netzwerkplanung in der Telekommunikation%
}{%
}{%
  Netzbetrieb und Netzplanung stecken bei Versorgungsunternehmen wie die Vodafone ganz tief im Kern
  der internen Sytemwelt~-- Gegenstand existentieller Prozesse und historisch tief verstrickt in
  Ablauf- und Aufbauorganisation.
  Wir berichten von der Integration der bestehenden GIS- und Dokumentationssystemen durch eine
  schlanke, Cloud-native Full-Stack-Open-Source-Anwendung für die Planung neuer Netzbestandteile,
  die alte Zöpfe nicht abschneidet, sondern zu einem E2E-Prozess verbindet.%
}%


%%%%%%%%%%%%%%%%%%%%%%%%%%%%%%%%%%%%%%%%%%%

% time: Thursday 14:00
% URL: https://pretalx.com/fossgis2023/talk/fossgis2023-24078-visualisierung-und-analyse-von-satellitenbildern-mit-der-enmap-box/

%

\noindent\abstractHSvier{%
  Andreas Janz, Benjamin Jakimow, Fabian Thiel, Patrick Hostert, Sebastian van der Linden%
}{%
  Visualisierung und Analyse von\linebreak Satellitenbildern mit der EnMAP-Box%
}{%
}{%
  Erdbeobachtungsdaten von Satellitenmissionen wie Landsat, Sentinel-2 und seit Kurzem auch
  hyperspektralen Missionen wie EnMAP stellen eine immer wichtigere Grundlage für raumbezogene
  Analysen dar. Unsere Demo-Session stellt die neuesten Möglichkeiten der EnMAP-Box vor, mir der in
  QGIS die Rasterdaten solcher Fernerkundungsmissionen schnell und effizient visualisiert und
  professionell analysiert werden können.%
}%


%%%%%%%%%%%%%%%%%%%%%%%%%%%%%%%%%%%%%%%%%%%

% time: Thursday 14:30
% URL: https://pretalx.com/fossgis2023/talk/fossgis2023-23736-die-etwas-andere-fahrradkarte/

%
\newSmallTimeslot{14:30}
\noindent\abstractHSeins{%
  Christopher Lorenz%
}{%
  Die etwas andere Fahrradkarte%
}{%
}{%
  Mit OpenMapTiles kann man Vectortiles erzeugen und die vorhandene Toolchain an seine Bedürfnisse
  anpassen. Darauf ist im Rahmen der Berliner OpenStreetMap-Verkehrswende-Gruppe eine Fahrradkarte
  zur Qualitätssicherung entstanden.%
}%
\pagebreak

%%%%%%%%%%%%%%%%%%%%%%%%%%%%%%%%%%%%%%%%%%%

% time: Thursday 14:30
% URL: https://pretalx.com/fossgis2023/talk/fossgis2023-23859-cloudoptimierte-formate-fr-kacheln-und-multidimensionale-rasterdaten/

%

\noindent\abstractHSzwei{%
  Marco Hugentobler%
}{%
  Cloud-optimierte Formate für Kacheln und multidimensionale Rasterdaten%
}{%
}{%
  Neben dem bereits etablierten Format COG für Cloud-optimized GeoTIFFs finden neue Formate für
  weitere Raster-Anwendungen immer mehr Interesse. Auf die Speicherung von $n$-dimensionalen
  Arraydaten, welche zum Beispiel in der Klimaforschung anfallen, ist das Zarr-Format ausgelegt.
  Für den Online- und Offlinegebrauch von Raster- und Vektorkacheln eignet sich das PMTiles-Format,
  dessen neuste Version 3 bereits Unterstützung in Leaflet und OpenLayers erhalten hat.%
}%


%%%%%%%%%%%%%%%%%%%%%%%%%%%%%%%%%%%%%%%%%%%

% time: Thursday 14:30
% URL: https://pretalx.com/fossgis2023/talk/fossgis2023-23417-nutzung-und-support-von-qgis-in-der-it-der-sachsenenergie/

%

\noindent\abstractHSdrei{%
  Christoph Jung%
}{%
  Nutzung und Support von QGIS in der IT der SachsenEnergie%
}{%
}{%
  QGIS bietet nicht nur Funktionen für Fachanwenderinnen und -anwender, sondern auch für die IT
  einer Organisation. Im Vortrag wird die Nutzung und der Support von QGIS durch die IT der
  SachsenEnergie präsentiert und ein Blick auf deren Geodateninfrastruktur hinsichtlich des
  Zusammenspiels proprietärer Systeme wie Smallworld und freier Software wie QGIS geworfen.%
}%


%%%%%%%%%%%%%%%%%%%%%%%%%%%%%%%%%%%%%%%%%%%

% time: Thursday 15:00
% URL: https://pretalx.com/fossgis2023/talk/fossgis2023-23582-radverkehrsatlas-openstreetmap-fr-die-schnelle-planung-von-radinfrastruktur/

%
\newTimeslot{15:00}
\noindent\abstractHSeins{%
  Tobias Jordans, Boris Hekele%
}{%
  Radverkehrsatlas~-- OpenStreetMap für die schnelle Planung von Radinfrastruktur%
}{%
}{%
  Der Radverkehrsatlas ist der Prototyp einer Webapplikation, über die Verwaltungen Zugang zu
  den Daten aus OpenStreetMap und der OSM-Community finden. Die OSM-Community hilft, die nötigen Daten
  anzureichern und zu vervollständigen. Die Verwaltung kann die Daten verifizieren. Dafür werden die
  Daten einheitlich aufbereitet und für die Anwendungsfälle der Planungsprozesse visualisiert.%
}%


%%%%%%%%%%%%%%%%%%%%%%%%%%%%%%%%%%%%%%%%%%%

% time: Thursday 15:00
% URL: https://pretalx.com/fossgis2023/talk/fossgis2023-23803-inspire-metadatensuche-in-qgis/

%

\noindent\abstractHSzwei{%
  Armin Retterath%
}{%
  INSPIRE-Metadatensuche in QGIS%
}{%
}{%
  Die Nutzer der Geoportale der Länder Hessen, Rheinland-Pfalz und Saarland können schon seit 2019
  direkt im jeweiligen Geoportal nach Daten und Diensten in Deutschland und Europa suchen.
  Seit Anfang 2022 steht diese Funktionalität nun auch allen Nutzern von QGIS zur Verfügung. Das neu
  entwickelte Plugin \emph{GeoPortal.rlp Metadata Search} ermöglicht erstmalig einen direkten und sehr
  einfachen Zugriff auf die Geodateninfrastrukturen in Deutschland und Europa.%
}%


%%%%%%%%%%%%%%%%%%%%%%%%%%%%%%%%%%%%%%%%%%%

% time: Thursday 15:00
% URL: https://pretalx.com/fossgis2023/talk/fossgis2023-23745-ein-meer-an-mglichkeiten-szenariobau-fr-das-offshore-stromnetz-in-qgis/

%

\noindent\abstractHSdrei{%
  Felix Jakob Fliegner%
}{%
  Ein Meer an Möglichkeiten~--\linebreak Szenariobau für das Offshore-Stromnetz in QGIS%
}{%
}{%
  Vorstellung eines Workflows zur Konsolidierung von Maritime-Spatial-Planning-Datensätzen der
  EU-Mitgliedsstaaden von Nord- und Ostsee in QGIS und QGIS Server und Visualisierung als Webkarte.
  Multikriterielle Analyse des Datensatzes zur Erstellung und Analyse eines Suchgraphen mit
  pgRouting, um automatisiert Szenarien für den Offshore-Windausbau der Zukunft für nachgelagerte
  Energiesystem-Modelle zu erstellen.%
}%


%%%%%%%%%%%%%%%%%%%%%%%%%%%%%%%%%%%%%%%%%%%

% time: Thursday 15:00
% URL: https://pretalx.com/fossgis2023/talk/fossgis2023-24079-analysis-ready-fernerkundungsdaten-erzeugen-mit-force/

%

\noindent\abstractHSvier{%
  David Frantz, Stefan Ernst, Benjamin Jakimow, Patrick Hostert%
}{%
  Analysis-ready Fernerkundungsdaten erzeugen mit FORCE%
}{%
}{%
  Frei verfügbare Landsat und Sentinel-2-Daten sind insbesondere seit Copernicus eine wichtige
  Grundlage für raumbezogene Analysen in Wirtschaft und Wissenschaft. Ihre Aufarbeitung für
  großräumige und sich über längere Zeiträume ersteckende Analysen stellt viele Anwender allerdings
  vor praktische Probleme. Dieser Vortrag zeigt, wie Landsat- und Sentinel-2-Daten mit FORCE für
  nationale bis kontinentale Analysen aufbereitet und effizient prozessiert werden können.%
}%


%%%%%%%%%%%%%%%%%%%%%%%%%%%%%%%%%%%%%%%%%%%

% time: Thursday 16:00
% URL: https://pretalx.com/fossgis2023/talk/fossgis2023-25510-geonode-anwendertreffen/

%
\newTimeslot{16:00}
\enlargethispage{3\baselineskip}

\noindent\abstractAnwBoF{%
  Florian Hoedt%
}{%
  GeoNode-Anwendertreffen%
}{%
}{%
  GeoNode ist eine webbasierte Anwendung und Plattform für den Einsatz in
  Geoforschungsdateninfrastrukturen.%
}%


%%%%%%%%%%%%%%%%%%%%%%%%%%%%%%%%%%%%%%%%%%%

% time: Thursday 16:00
% URL: https://pretalx.com/fossgis2023/talk/fossgis2023-23855-ask-me-anything-qgis-/

%

\noindent\abstractExpBoFAnw{%
  Matthias Kuhn, Marco Bernasocchi%
}{%
  Ask me anything QGIS!%
}{%
}{%
  QGIS-Chairman Marco Bernasocchi und Kernentwickler Matthias Kuhn stehen während einer Stunde für
  alle QGIS-relevanten Fragen zur Verfügung.%
}%


%%%%%%%%%%%%%%%%%%%%%%%%%%%%%%%%%%%%%%%%%%%

% time: Thursday 16:00
% URL: https://pretalx.com/fossgis2023/talk/fossgis2023-23833-open-data-zu-wasserbezogenen-klimarisiken-wo-steht-berlin-brandenburg-/

%

\noindent\abstractHSeins{%
  Fabio Brill, Tobia Lakes%
}{%
  Open Data zu wasserbezogenen Klimarisiken: Wo steht Berlin-Brandenburg?%
}{%
}{%
  Frei zugängliche Daten zu wasserbezogenen Klimarisiken sind Bausteine einer resiliente
  Gesellschaft. Im Rahmen des Vortrags werden bestehende Daten und Informationsplattformen aus
  Ber\-lin-Brandenburg in Hinblick auf Inhalt, Format, Zugänglichkeit, Dokumentation und Nutzbarkeit
  evaluiert.%
}%
\pagebreak

%%%%%%%%%%%%%%%%%%%%%%%%%%%%%%%%%%%%%%%%%%%

% time: Thursday 16:00
% URL: https://pretalx.com/fossgis2023/talk/fossgis2023-23695-geodatenanalyse-in-der-cloud-mit-ogc-api-processes-und-pygeoapi/

%

\noindent\abstractHSzwei{%
  Hannes Blitza, Christian Mayer%
}{%
  Geodatenanalyse in der Cloud mit\linebreak OGC API Processes und pygeoapi%
}{%
}{%
  Vor dem Hintergrund der stetig wachsenden Menge an frei verfügbaren Daten gewinnt der Einsatz von
  standardisierten cloudbasierten Tools zur Daten-Prozessierung und -Analyse zunehmend an Bedeutung.
  Im Forschungsprojekt KLIPS werden auf Basis des neuen OGC API Processes Demonstratoren entwickelt,
  die Rasterdaten zu urbanen Hitzeinseln analysieren. Die Schnittstelle wird mit mit pygeoapi
  aufgesetzt, für die Analyse kommen bewährte Algorithmen von GRASS und GDAL zum Einsatz.%
}%


%%%%%%%%%%%%%%%%%%%%%%%%%%%%%%%%%%%%%%%%%%%

% time: Thursday 16:00
% URL: https://pretalx.com/fossgis2023/talk/fossgis2023-23730-osm-daten-und-indoor-karten-in-kde-itinerary/

%

\noindent\abstractHSdrei{%
  Volker Krause%
}{%
  OSM-Daten und Indoor-Karten in KDE Itinerary%
}{%
}{%
  KDE's freie Reise-App Itinerary nutzt OSM Daten auf vielfältige Art und Weise, beispielsweise für
  Indoor-Karten von Bahnhöfen und Flughäfen. Dieser Vortrag betrachtet die dabei entstandenen
  Lösungen und die angetroffenen Herausforderungen.%
}%
\pagebreak

%%%%%%%%%%%%%%%%%%%%%%%%%%%%%%%%%%%%%%%%%%%

% time: Thursday 16:00
% URL: https://pretalx.com/fossgis2023/talk/fossgis2023-23779-alkis-nas-daten-in-qgis-und-im-webgis-qgis-server-mit-lizmap-nutzen/

%

\noindent\abstractHSvier{%
  Günter Wagner%
}{%
  ALKIS-NAS-Daten in QGIS und im WebGIS (QGIS-Server mit Lizmap) nutzen%
}{%
}{%
  Diese Demo-Session zeigt die Nutzung der ALKIS-NAS-Daten über die Tools der PostNAS-Suite in QGIS
  und im WebGIS (mit QGIS-Server und dem Lizmap-Client).
  Und richtet sich insbesondere an Anwender, die bisher nicht oder nur wenig mit der
  PostgreSQL-/PostGIS-Datenbank gearbeitet haben.%
}%


%%%%%%%%%%%%%%%%%%%%%%%%%%%%%%%%%%%%%%%%%%%

% time: Thursday 16:30
% URL: https://pretalx.com/fossgis2023/talk/fossgis2023-23825-open-data-open-source-open-berlin/

%
\newSmallTimeslot{16:30}
\noindent\abstractHSeins{%
  Lisa Stubert%
}{%
  Open Data, Open Source, Open Berlin%
}{%
}{%
  Von Open Data und Open Source kann die gesamte Stadtgesellschaft profitieren. Sie baut Wissen auf,
  erfährt Erleichterungen z.\,B. in der Mobilität oder bei der Inanspruchnahme von Dienstleistungen
  und nutzt Transparenz und Partizipationsmöglichkeiten bei politischen Entscheidungen. Welche
  spannenden Datenanwendungen und Werkzeuge können auf offenen Daten aufbauen? Wir zeigen aktuelle
  Beispiele aus Berlin.%
}%
\pagebreak

%%%%%%%%%%%%%%%%%%%%%%%%%%%%%%%%%%%%%%%%%%%

% time: Thursday 16:30
% URL: https://pretalx.com/fossgis2023/talk/fossgis2023-23842-eo-lab-shogun-webgis-actinia-rasterprozessierung-in-der-cloud/

%

\noindent\abstractHSzwei{%
  Arnulf Christl%
}{%
  EO-Lab: SHOGun WebGIS, actinia-Rasterprozessierung in der Cloud%
}{%
}{%
  EO-Lab ist eine neue Cloud, die einen einfachen Zugang zu Satellitendaten anbietet. Der
  Schwerpunkt liegt auf nationalen Satelitendaten wie EnMAP, TerraSar-X und TanDEM-X, wie auch Daten
  der Sentinel-Satelliten. In diesem Vortrag stellen wir die verwendeten Komponenten des Web-GIS-Client,
  SHOGun, openEO-backend, actinia und GRASS GIS und deren Zusammenspiel in einer Cloud-Architektur vor.%
}%


%%%%%%%%%%%%%%%%%%%%%%%%%%%%%%%%%%%%%%%%%%%

% time: Thursday 16:30
% URL: https://pretalx.com/fossgis2023/talk/fossgis2023-23912-floor-plan-extraction-from-digital-building-models/

%

\noindent\abstractHSdrei{%
  Helga Tauscher, Subhashini Krishnakumar%
}{%
  Floor plan extraction from digital\linebreak building models%
}{%
}{%
  Interior building data is gaining popularity, but while there is an abundance of exterior data for
  navigation and other purposes, the  availability of indoor data is meagre. This paper presents how
  simplified indoor data can be extracted from digital building models, which are rich in interior
  information, and can be converted to formats such as CityGML, IndoorGML, and OpenstreetMap to
  increase the availability of indoor data.%
}%


%%%%%%%%%%%%%%%%%%%%%%%%%%%%%%%%%%%%%%%%%%%

% time: Thursday 17:00
% URL: https://pretalx.com/fossgis2023/talk/fossgis2023-23864-foss-gis-in-der-berliner-verwaltung-ein-erfolgsmodell-/

%
\newTimeslot{17:00}
\noindent\abstractHSeins{%
  Enrico Stein, Matthias Schroeder, Manuel Gehlhoff%
}{%
  FOSS-GIS in der Berliner Verwaltung~-- ein Erfolgsmodell?%
}{%
}{%
  FOSS-GIS in der Berliner Verwaltung, funktioniert das? Dazu soll der Vortrag einen Einblick in die
  gemachten Erfahrungen und die möglichen Einsatzfelder geben. Im Fokus stehen zwei Senatsverwaltungen
  (SenUMVK und SenSBW). Der Vortrag soll zu einer weiteren Diskussion und einer Vernetzung anregen.%
}%


%%%%%%%%%%%%%%%%%%%%%%%%%%%%%%%%%%%%%%%%%%%

% time: Thursday 17:00
% URL: https://pretalx.com/fossgis2023/talk/fossgis2023-23814-geodatenverarbeitung-mit-workflow-engines/

%

\noindent\abstractHSzwei{%
  Pirmin Kalberer%
}{%
  Geodatenverarbeitung mit Workflow-Engines%
}{%
}{%
  Workflow-Engines wie Apache Airflow sind ein wichtiges Instrument im Bereich Data Science. Sie
  bieten die Infrastruktur zum Definieren, Ausführen und Überwachen einer Abfolge von Schritten
  eines Datenverarbeitungsprozesses. Dieser Vortrag vergleicht eine Auswahl verfügbarer Open-Source-Workflow-Engines,
  die sich besonders für Workflows mit Geodatenverarbeitung eignen. Zudem wird der
  Standard OGC API Processes, ein REST-API zur Ausführung und Überwachung von Prozessen
  vorgestellt.%
}%
\newpage
%%%%%%%%%%%%%%%%%%%%%%%%%%%%%%%%%%%%%%%%%%%

% time: Thursday 17:00
% URL: https://pretalx.com/fossgis2023/talk/fossgis2023-23816-wo-bin-ich-lsen-von-fuzzy-wobbling-geo-locations-mit-dem-qgis-plugin-crs-guesser-/

%

\noindent\abstractHSdrei{%
  Brigit Danthine, Florian Thiery%
}{%
  Wo bin ich? Lösen von fuzzy wobbling Geo-Locations mit dem QGIS-Plugin CRS Guesser%
}{%
}{%
  Das Plugin \emph{CRS Guesser} für QGIS ermöglicht es, eine Koordinate oder einen Layer mit einem
  unbekannten Koordinatensystem einzugeben und automatisch eine Liste möglicher Koordinatensysteme
  zu durchsuchen. Das Plugin verbindet zudem die klassische Geowelt mit der des Semantic Web mit
  Nutzung von Ressourcen in Wikidata. Dieser Lightning Talk gibt einen Einblick in das Plugin.%
}%


%%%%%%%%%%%%%%%%%%%%%%%%%%%%%%%%%%%%%%%%%%%

% time: Thursday 17:05
% URL: https://pretalx.com/fossgis2023/talk/fossgis2023-23749-rumliche-layerfilter-fr-effiziente-re-s-arbeiten-ein-neues-qgis-plugin/

%
\newSmallTimeslot{17:05}
\noindent\abstractHSdrei{%
  Johannes Kröger, Mathias Gröbe%
}{%
  Räumliche Layerfilter für effiziente(re)s Arbeiten: Ein neues QGIS-Plugin%
}{%
}{%
  Ein neues QGIS-Plugin ermöglicht es, die Layer eines Projektes schnell und einfach auf ein
  bestimmtes räumliches Gebiet zu beschränken. Damit kann die Arbeit stark beschleunigt werden, denn
  es werden keine unnötigen Daten von der Datenquelle abgefragt und QGIS muss weniger Daten in der
  Karte zeichnen. So ist es unter anderem möglich, jeweils im Kontext unterschiedlicher Gemeinden
  oder Gitterzellen zu arbeiten.%
}%


%%%%%%%%%%%%%%%%%%%%%%%%%%%%%%%%%%%%%%%%%%%

% time: Thursday 17:10
% URL: https://pretalx.com/fossgis2023/talk/fossgis2023-23845-manahmen-zur-gewsserunterhaltung-managen-mit-qgis-postgis-und-qfield/

%
\newTimeslot{17:10}
\noindent\abstractHSdrei{%
  Klaus Mithöfer%
}{%
  Maßnahmen zur Gewässerunterhaltung managen mit QGIS, PostGIS und QField%
}{%
}{%
  Der Unterhaltungsverband "`Mittlere Hase"' betreut  insgesamt 1\,400 Gewässerkilometer für 700
  Kunden. Für die Maßnahmenplanung und Erfassung von Arbeitsständen werden neben QGIS eine PostGIS-Datenbank
  und mobile Endgeräte mit der Software QField eingesetzt. Da im Gelände offline
  gearbeitet werden muss werden zentrale Daten täglich synchronisiert. Technisch möglich ist die
  Synchronisation durch Funktionen der GBD Websuite und den Einsatz einer eigens entwickelten App
  zum Datentransfer.%
}%


%%%%%%%%%%%%%%%%%%%%%%%%%%%%%%%%%%%%%%%%%%%

% time: Thursday 17:15
% URL: https://pretalx.com/fossgis2023/talk/fossgis2023-23727-das-neue-geodatenportal-der-marinen-dateninfrastruktur-deutschland-mdi-de-/

%
\newSmallTimeslot{17:15}
\noindent\abstractHSdrei{%
  Hannes Blitza, Johannes Melles%
}{%
  Das neue Geodatenportal der Marinen Dateninfrastruktur Deutschland%
}{%
}{%
  Im Rahmen der Marinen Dateninfrastruktur Deutschland (MDI-DE) wurde ein nationales Meeres- und
  Küsteninformationssystem eingerichtet. Die MDI-DE stellt die Datenbestände der wichtigsten
  Datenhalter an der Küste interoperabel bereit.
  Fokus des Vortrags ist das neu entwickelte MDI-DE-Geoportal, das basierend auf
  Open-Source-Komponenten, vorrangig dem Masterportal, das gezielte Auffinden, Analysieren und
  Exportieren der marinen Daten ermöglicht.%
}%


%%%%%%%%%%%%%%%%%%%%%%%%%%%%%%%%%%%%%%%%%%%

% time: Thursday 17:25
% URL: https://pretalx.com/fossgis2023/talk/fossgis2023-27095-spontane-lightning-talks/

%
\newSmallTimeslot{17:25}
\noindent\abstractHSdrei{%
}{%
  Spontane Lightning Talks%
}{%
}{%
  Spontan eingereichte Lightning Talks. Jeder kann einen Vortrag halten, die Registrierung erfolgt am
  FOSSGIS-Stand.%
}%


%%%%%%%%%%%%%%%%%%%%%%%%%%%%%%%%%%%%%%%%%%%

% time: Thursday 17:45
% URL: https://pretalx.com/fossgis2023/talk/fossgis2023-23886-der-schnelle-weg-in-die-digitale-souvernitt-ffentliche-ausschreibungen-mit-foss/

%
\newSmallTimeslot{17:45}
\noindent\abstractHSeins{%
  Podiumsdikussion%
}{%
  Der schnelle Weg in die digitale Souveränität~-- öffentliche Ausschreibungen mit FOSS%
}{%
}{%
  Seit Juni 2021 beschäftigt sich die Arbeitsgruppe \emph{Öffentliche Ausschreibungen mit FOSS} des
  FOSSGIS e.\,V.  mit dem Thema der Beschaffung und Vergabe von IT-Lösungen auf Basis von FOSS. In
  dieser Dialogrunde wollen wir mit Vertreter:innen, die an Ausschreibungsverfahren und der
  anschließenden Entwicklung und Implementierung beteiligt sind, die häufig vorkommenden Frage- und
  Problemstellungen erörtern sowie Tipps für das Gelingen geben.%
}%


%%%%%%%%%%%%%%%%%%%%%%%%%%%%%%%%%%%%%%%%%%%

% time: Thursday 19:00
% URL: https://pretalx.com/fossgis2023/talk/fossgis2023-25509-mitgliederversammlung/

%
\newSmallTimeslot{19:00}
\noindent\abstractHSdrei{%
}{%
  Mitgliederversammlung\linebreak FOSSGIS e.\,V.%
}{%
}{%
  Zur jährlich stattfindenden Versammlung des FOSSGIS e.\,V. sind alle Mitglieder herzlich eingeladen,
  teilzunehmen und sich zu beteiligen.  Der FOSSGIS e.\,V. lädt ein zum Kennenlernen, zur Diskussion,
  Abstimmung und Wahlen.%
}%


%%%%%%%%%%%%%%%%%%%%%%%%%%%%%%%%%%%%%%%%%%%
